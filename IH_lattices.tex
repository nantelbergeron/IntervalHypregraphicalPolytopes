\documentclass[reqno]{amsart}

\usepackage{enumerate, amsmath, amsfonts, amssymb, amsthm, wasysym, graphics, graphicx, xcolor, url, hyperref, hypcap, a4wide, pdflscape, multido, xargs, colortbl, multicol, multirow, calc, shuffle, hvfloat}
\hypersetup{colorlinks=true, citecolor=PineGreen, linkcolor=PineGreen}
%\usepackage[all]{xy}
\usepackage{tikz}\usetikzlibrary{trees,snakes,shapes,arrows,matrix,calc}
\usepackage{comment}
\usepackage{etex}
\usepackage{ulem}\normalem % to strike through a word
\usepackage[noabbrev,capitalise]{cleveref}
\setlength{\abovecaptionskip}{10pt}
\setlength{\belowcaptionskip}{5pt}

%\reserveinserts{50}
\graphicspath{{figures/}}
\makeatletter
\def\input@path{{figures/}}
\makeatother

%%%%%%%%%%%%%%%%%%%%%%%%%%%%%%%%%%%%%%

\title{Lattices of Segment Hypergraphic Polytopes}

\author[N.~Bergeron]{Nantel Bergeron} 
\address[N.~Bergeron]{Department of Mathematics and Statistics, York University, Toronto}
\email{bergeron@yorku.ca}
\urladdr{http://bergeron.mathstats.yorku.ca}

\author[V.~Pilaud]{Vincent Pilaud}
\address[V.~Pilaud]{??? CNRS \& LIX, \'Ecole Polytechnique, Palaiseau}
\email{vincent.pilaud@lix.polytechnique.fr}
\urladdr{http://www.lix.polytechnique.fr/~pilaud/}

\thanks{
NB was supported by NSERC and York Research Chair in Applied Algebra.
VP ???).
}



%%%%%%%%%%%%%%%%%%%%%%%%%%%%%%%%%%%%%%

% theorems
\newtheorem{theorem}{Theorem}[section]
\newtheorem{theoremA}{Theorem}
\renewcommand{\thetheoremA}{\Alph{theoremA}}
\newtheorem{corollary}[theorem]{Corollary}
\newtheorem{proposition}[theorem]{Proposition}
\newtheorem{lemma}[theorem]{Lemma}
\newtheorem{conjecture}[theorem]{Conjecture}
\crefname{conjecture}{Conjecture}{Conjectures}
\newtheorem{conjectureA}{Conjecture}
\renewcommand{\theconjectureA}{\Alph{conjectureA}}
\crefname{conjectureA}{Conjecture}{Conjectures}

\theoremstyle{definition}
\newtheorem{definition}[theorem]{Definition}
\newtheorem{example}[theorem]{Example}
\newtheorem{remark}[theorem]{Remark}
\newtheorem{question}[theorem]{Question}
\newtheorem{notation}[theorem]{Notation}
\newtheorem{openproblem}[theorem]{Open problem}


% newcommands
% math special letters
\newcommand{\R}{\mathbb{R}} % reals
\newcommand{\N}{\mathbb{N}} % naturals
\newcommand{\Z}{\mathbb{Z}} % integers
\newcommand{\I}{\mathbb{I}} % set of integers
\newcommand{\C}{\mathbb{C}} % set of summands
\renewcommand{\b}[1]{\boldsymbol{#1}} % bold
\newcommand{\cal}[1]{\mathcal{#1}} % cal

% math commands
\newcommand{\set}[2]{\left\{ #1 \;\middle|\; #2 \right\}} % set notation
\newcommand{\bigset}[2]{\big\{ #1 \;|\; #2 \big\}} % big set notation
\newcommand{\biggset}[2]{\bigg\{ #1 \;\bigg|\; #2 \bigg\}} % big set notation
\newcommand{\multiset}[2]{\left\{\!\!\left\{ #1 \;\middle|\; #2 \right\}\!\!\right\}} % multiset notation
\newcommand{\bigmultiset}[2]{\big\{\!\!\big\{ #1 \;|\; #2 \big\}\!\!\big\}} % big multiset notation
\newcommand{\ssm}{\smallsetminus} % small set minus
\newcommand{\dotprod}[2]{\langle #1 | #2 \rangle} % dot product
\newcommand{\symdif}{\triangle} % symmetric difference
\newcommand{\one}{{1\!\!1}} % the all one vector
\newcommand{\eqdef}{\mbox{\,\raisebox{0.2ex}{\scriptsize\ensuremath{\mathrm:}}\ensuremath{=}\,}} % :=
\newcommand{\defeq}{\mbox{~\ensuremath{=}\raisebox{0.2ex}{\scriptsize\ensuremath{\mathrm:}} }} % =:
\newcommand{\polar}{^\diamond} % polar
\newcommand{\simplex}{\triangle} % simplex

% operators
\DeclareMathOperator{\conv}{conv} % convex hull
\DeclareMathOperator{\cone}{cone} % cone hull
\DeclareMathOperator{\arr}{Arr} % arrangements
\DeclareMathOperator{\Inv}{Inv} % inversion set
\DeclareMathOperator{\Ninv}{Ninv} % non-inversion set
\DeclareMathOperator{\DemazureProduct}{Dem} % non-inversion set

% others
\newcommand{\fix}[1]{{\bf FIXME: }#1} % emphasis of a problem to FIX
\newcommand{\ie}{\textit{i.e.}~} % id est
\newcommand{\eg}{\textit{e.g.}~} % exempli gratia
\newcommand{\Eg}{\textit{E.g.}~} % exempli gratia
\newcommand{\aka}{\textit{aka.}~} % also known as
\newcommand{\viceversa}{\textit{vice versa}} % vice versa
\newcommand{\ordinal}{\textsuperscript{th}} % th for ordinals
\newcommand{\ex}[1]{^{\textrm{ex#1}}} % example
\newcommand{\para}[1]{\medskip\noindent\textbf{#1}} % paragraph
\newcommand{\subpara}[1]{\smallskip\noindent\textit{#1.}} % paragraph
\definecolor{PineGreen}{RGB}{2,120,120} % pinegreen color
\definecolor{darkgreen}{RGB}{57,181,74} % darkgreen color
\newcommand{\blue}[1]{{\color{blue} #1}} % blue
\newcommand{\red}[1]{{\color{red} #1}} % red
\newcommand{\green}[1]{{\color{darkgreen} #1}} % green
\newcommand{\defn}[1]{\textbf{\textsf{\color{PineGreen} #1}}} % emphasis of a definition
\usepackage{todonotes}
\newcommand{\nantel}[1]{\todo[color=red!30]{#1 \\ \hfill --- N.}}
\newcommand{\Nantel}[1]{\todo[inline,color=red!30]{#1 \\ \hfill --- N.}}
\newcommand{\vincent}[1]{\todo[color=blue!30]{#1 \\ \hfill --- V.}}
\newcommand{\Vincent}[1]{\todo[inline,color=blue!30]{#1 \\ \hfill --- V.}}

% permutations
\newcommand{\fS}{\mathfrak{S}} % symmetric group
\newcommand{\fR}{\mathfrak{R}} % subset symmetric group


% lattices
\newcommand{\meet}{\wedge} % meet
\newcommand{\join}{\vee} % join
\newcommand{\less}{\vartriangleleft} % smaller WOIP
\newcommand{\more}{\vartriangleright} % larger WOIP
\newcommand{\contactLess}[1]{\less_{#1}} % smaller contact graph
\newcommand{\contactMore}[1]{\more_{#1}} % larger contact graph
\newcommand{\projDown}{\pi_\downarrow} % Down projection
\newcommand{\projUp}{\pi^\uparrow} % Down projection

% Orientation, Hypergraph and Segments
\newcommand{\Or}{\mathcal O}  % acyclic orientation for hypergraph (vertices for us)
\newcommand{\HH}{\mathbb H}  % general hypergraph
\newcommand{\II}{\mathbb I} % segment hypergraph


%%%%%%%%%%%%%%%%%%%%%%%%%%%%%%%%%%%%%%
%%%%%%%%%%%%%%%%%%%%%%%%%%%%%%%%%%%%%%
%%%%%%%%%%%%%%%%%%%%%%%%%%%%%%%%%%%%%%

\begin{document}

\begin{abstract}
	Lets be real, not just abstract
\end{abstract}

\vspace*{-.8cm}

\maketitle
\nantel{Todo ``Nantel" avec majuscule sont en ligne avec le texte, "nantel" est comme avant}
\Nantel{Vincent, verifie ton affiliation, grant et addresse}


\tableofcontents

%%%%%%%%%%%%%%%%%%%%%%%%%%%%%%%%%%%%%%
%%%%%%%%%%%%%%%%%%%%%%%%%%%%%%%%%%%%%%
%%%%%%%%%%%%%%%%%%%%%%%%%%%%%%%%%%%%%%

\section{Introduction}
\label{sec:introduction}


%%%%%%%%%%%%%%%%%%%%%%%%%%%%%%%%%%%%%%
%%%%%%%%%%%%%%%%%%%%%%%%%%%%%%%%%%%%%%
%%%%%%%%%%%%%%%%%%%%%%%%%%%%%%%%%%%%%%

\section{Hypergraphical Polytope and its partial order}
\label{sec:IHP}


%%%%%%%%%%%%%%%%%%%%%%%%%%%%%%%%%%%%%%
\subsection{Hypergraphic Polytope} 
\label{subsec:D_H}

Let $[n]=\{1,2,\ldots,n\}$. A \defn{hypergraph} $\HH$ on $[n]$, is a collection of  subsets of $[n]$.
By convention, we will assume that for all $i\in [n]$, we have $\{i\}\in \HH$.
Following~\cite[Def 2.11]{BenBerMac}, given an hypergraph $\HH$, we define the \defn{hypergraphic polytope}
$\Delta_{\HH}$ as the the Minkowski sum
 $$\Delta_{\HH} = \sum_{H\in \HH} \Delta_H\,,$$
where $\Delta_H$ is the simplex given by the convex hull of the points $\{e_i | i\in H\}\subset \R^n$.

\begin{example}\label{ex:DH}
For the hypergraph
$\HH=\big\{\{1\},\{2\},\{3\},\{4\},\{3,4\},\{1,2,3\}\big\}$,
we  have
$$\begin{array}{ccc}
 \begin{tikzpicture}[scale=1,baseline=.5cm]
	\node (1) at (0.6,1.0) {$\scriptstyle e_1$};
	\node (2) at (-.2,-.2) {$\scriptstyle e_2$};
	\node (3) at (1.5,-.2) {$\scriptstyle e_3$};
	\draw [fill=blue!40] (0,0) -- (.6,.75) -- (1.2,0) --(0,0) ; 
\end{tikzpicture} \quad &
 \begin{tikzpicture}[scale=1,baseline=.5cm]
	\node at (-.2,0) {$\scriptstyle e_3$};
	\node at (1.2,.5) {$\scriptstyle e_4$};
	\draw [thick,color=red] (0,0) -- (1,.5 ); 
\end{tikzpicture} \quad &
\begin{tikzpicture}[scale=1,baseline=.5cm]
	\draw [fill=gray!10] (0,0) -- (.6,.75)-- (1.6,1.25) -- (2.2,.5) -- (1.2,0) --(0,0) ; 
	\draw [color=gray!10,fill=blue!20] (0,0) -- (.6,.75) -- (1.2,0) --(0,0) ; 
	\draw [dotted,color=red] (0,0)--(1,.5);
	\draw [dotted,color=blue] (1.6,1.25)--(1,.5)--(2.2,.5);
	\draw (0,0) -- (.6,.75)-- (1.6,1.25) -- (2.2,.5) -- (1.2,0) -- (0,0) ; 
	\draw [thick,color=red!80] (1.2,0) -- (2.2,0.5); 
\end{tikzpicture}\\
\blue{{ \Delta}_{123}}& \red{{ \Delta}_{34}} & \Delta_{\HH}=\Delta_1+\Delta_2+\Delta_3+\Delta_4+\blue{{ \Delta}_{123}}+ \red{{ \Delta}_{34}}\\
\end{array}
$$
which is a 3-dimensional polytope sitting in $\R^4$. 
\end{example}

\begin{remark}\label{rem:single} In~\cite{BenBerMac}, the authors always assume that for all $i\in[n]$ we have $\{i\}\not\in\HH$ whereas here we took the opposite convention. 
When quoting result from~\cite{BenBerMac}, the reader will have to be mindful of this difference.
We point out, as in  Example~\ref{ex:DH}, that the $\Delta_i$ are single points and in the definition of $\Delta_{\HH}$ it simply translate the polytope in the direction of $e_i$ but does not affect the  face structure of the polytope.
\end{remark}
%%%%%%%%%%%%%%%%%%%%%%%%%%%%%%%%%%%%%%
\subsection{The Poset of a Hypergraphic Polytope} 
\label{subsec:P_H}
One of the main result of~\cite[Thm 3.18]{BenBerMac} is to identify the faces of $\Delta_\HH$ with the acyclic orientations of certain quotient of $\HH$.  A useful tool,~\cite[Lem 2.9]{BenBerMac}, gives us a surjective map $\Omega$ from the set composition (ordered set partitions) of $[n]$ to the set of faces of $\Delta_\HH$. 
 In the present work, we will be interested in the skeleton (faces of dimension 1 and 2)
of the polytope as an oriented graph $P_{\HH}$. It will turn out that $P_{\HH}$ is a poset (partial order).  The map $\Omega$ identify the vertices $V_{\HH}$ of $\Delta_{\HH}$ as the image of the set compositions of $[n]$ with parts all of size 1.  In turn, this can be identified as ordered list of $[n]$ corresponding to $\fS_n$ the permutations of the set $[n]$. This gives us a map
\begin{equation}\Omega_1 \colon \fS_n \to V_{\HH}\,.
\end{equation} 
More explicitly, given $\pi\in\fS_n$ we construct an acyclic orientation $\Or_\pi$ of $\HH$ as follow. For any $H\in \HH$ such that $|H|>1$ we orient $H$  with the source $\{a\}$, where $a$ is the leftmost element of $H$ in $\pi$. That is
	$$\Or_\pi = \Big\{\big(\{a\},H\ssm\{a\}\big)\big| H\in \HH, a=\pi\big(\min\{j: \pi(j)\in H\}\big)\Big\}\,.
	$$
We immediately obtain the following lemma.
\begin{lemma}\label{lem:vertices}
For any hypergraph $\HH$, the set of vertices of $\Delta_{\HH}$ is given by $V_{\HH}=\{\Or_\pi: \pi\in \fS_n\}$.
\end{lemma}

\Nantel{I will then introduce a map $\Omega_2$ that defines the (oriented) edges of $P_{\HH}$. That will give us a well define partial order. It is in fact a surjective map from the weak order to $P_{\HH}$}
%%%%%%%%%%%%%%%%%%%%%%%%%%%%%%%%%%%%%%
%%%%%%%%%%%%%%%%%%%%%%%%%%%%%%%%%%%%%%
%%%%%%%%%%%%%%%%%%%%%%%%%%%%%%%%%%%%%%

\section*{Acknowledgments}

Thanks!
%%%%%%%%%%%%%%%%%%%%%%%%%%%%%%%%%%%%%%
%%%%%%%%%%%%%%%%%%%%%%%%%%%%%%%%%%%%%%
%%%%%%%%%%%%%%%%%%%%%%%%%%%%%%%%%%%%%%

\bibliographystyle{alpha}
\bibliography{IH_lattices}
\label{sec:biblio}

%%%%%%%%%%%%%%%%%%%%%%%%%%%%%%%%%%%%%%

\end{document}
