\documentclass[reqno]{amsart}

\usepackage{enumerate, amsmath, amsfonts, amssymb, amsthm, wasysym, graphics, graphicx, xcolor, url, hyperref, hypcap, a4wide, pdflscape, multido, xargs, colortbl, multicol, multirow, calc, shuffle, hvfloat}
\hypersetup{colorlinks=true, citecolor=PineGreen, linkcolor=PineGreen}
%\usepackage[all]{xy}
\usepackage{tikz}\usetikzlibrary{trees,snakes,shapes,arrows,matrix,calc,arrows.meta}
\usepackage{comment}
\usepackage{etex}
\usepackage{ulem}\normalem % to strike through a word
\usepackage[noabbrev,capitalise]{cleveref}
\setlength{\abovecaptionskip}{10pt}
\setlength{\belowcaptionskip}{5pt}

%\reserveinserts{50}
\graphicspath{{figures/}}
\makeatletter
\def\input@path{{figures/}}
\makeatother

%%%%%%%%%%%%%%%%%%%%%%%%%%%%%%%%%%%%%%

\title{Lattices of Segment Hypergraphic Polytopes}

\author[N.~Bergeron]{Nantel Bergeron} 
\address[N.~Bergeron]{Department of Mathematics and Statistics, York University, Toronto}
\email{bergeron@yorku.ca}
\urladdr{http://bergeron.mathstats.yorku.ca}

\author[V.~Pilaud]{Vincent Pilaud}
\address[V.~Pilaud]{??? CNRS \& LIX, \'Ecole Polytechnique, Palaiseau}
\email{vincent.pilaud@lix.polytechnique.fr}
\urladdr{http://www.lix.polytechnique.fr/~pilaud/}

\thanks{
NB was supported by NSERC and York Research Chair in Applied Algebra.
VP ???).
}



%%%%%%%%%%%%%%%%%%%%%%%%%%%%%%%%%%%%%%

% theorems
\newtheorem{theorem}{Theorem}[section]
\newtheorem{theoremA}{Theorem}
\renewcommand{\thetheoremA}{\Alph{theoremA}}
\newtheorem{corollary}[theorem]{Corollary}
\newtheorem{proposition}[theorem]{Proposition}
\newtheorem{lemma}[theorem]{Lemma}
\newtheorem{conjecture}[theorem]{Conjecture}
\crefname{conjecture}{Conjecture}{Conjectures}
\newtheorem{conjectureA}{Conjecture}
\renewcommand{\theconjectureA}{\Alph{conjectureA}}
\crefname{conjectureA}{Conjecture}{Conjectures}

\theoremstyle{definition}
\newtheorem{definition}[theorem]{Definition}
\newtheorem{example}[theorem]{Example}
\newtheorem{remark}[theorem]{Remark}
\newtheorem{question}[theorem]{Question}
\newtheorem{notation}[theorem]{Notation}
\newtheorem{openproblem}[theorem]{Open problem}


% newcommands
% math special letters
\newcommand{\R}{\mathbb{R}} % reals
\newcommand{\N}{\mathbb{N}} % naturals
\newcommand{\Z}{\mathbb{Z}} % integers
\newcommand{\I}{\mathbb{I}} % set of integers
\newcommand{\C}{\mathbb{C}} % set of summands
\renewcommand{\b}[1]{\boldsymbol{#1}} % bold
\newcommand{\cal}[1]{\mathcal{#1}} % cal

% math commands
\newcommand{\set}[2]{\left\{ #1 \;\middle|\; #2 \right\}} % set notation
\newcommand{\bigset}[2]{\big\{ #1 \;|\; #2 \big\}} % big set notation
\newcommand{\biggset}[2]{\bigg\{ #1 \;\bigg|\; #2 \bigg\}} % big set notation
\newcommand{\multiset}[2]{\left\{\!\!\left\{ #1 \;\middle|\; #2 \right\}\!\!\right\}} % multiset notation
\newcommand{\bigmultiset}[2]{\big\{\!\!\big\{ #1 \;|\; #2 \big\}\!\!\big\}} % big multiset notation
\newcommand{\ssm}{\smallsetminus} % small set minus
\newcommand{\dotprod}[2]{\langle #1 | #2 \rangle} % dot product
\newcommand{\symdif}{\triangle} % symmetric difference
\newcommand{\one}{{1\!\!1}} % the all one vector
\newcommand{\eqdef}{\mbox{\,\raisebox{0.2ex}{\scriptsize\ensuremath{\mathrm:}}\ensuremath{=}\,}} % :=
\newcommand{\defeq}{\mbox{~\ensuremath{=}\raisebox{0.2ex}{\scriptsize\ensuremath{\mathrm:}} }} % =:
\newcommand{\polar}{^\diamond} % polar
\newcommand{\simplex}{\triangle} % simplex

% operators
\DeclareMathOperator{\conv}{conv} % convex hull
\DeclareMathOperator{\cone}{cone} % cone hull
\DeclareMathOperator{\arr}{Arr} % arrangements
\DeclareMathOperator{\Inv}{Inv} % inversion set
\DeclareMathOperator{\Ninv}{Ninv} % non-inversion set
\DeclareMathOperator{\DemazureProduct}{Dem} % non-inversion set

% others
\newcommand{\fix}[1]{{\bf FIXME: }#1} % emphasis of a problem to FIX
\newcommand{\ie}{\textit{i.e.}~} % id est
\newcommand{\eg}{\textit{e.g.}~} % exempli gratia
\newcommand{\Eg}{\textit{E.g.}~} % exempli gratia
\newcommand{\aka}{\textit{aka.}~} % also known as
\newcommand{\viceversa}{\textit{vice versa}} % vice versa
\newcommand{\ordinal}{\textsuperscript{th}} % th for ordinals
\newcommand{\ex}[1]{^{\textrm{ex#1}}} % example
\newcommand{\para}[1]{\medskip\noindent\textbf{#1}} % paragraph
\newcommand{\subpara}[1]{\smallskip\noindent\textit{#1.}} % paragraph
\definecolor{PineGreen}{RGB}{2,120,120} % pinegreen color
\definecolor{darkgreen}{RGB}{57,181,74} % darkgreen color
\newcommand{\blue}[1]{{\color{blue} #1}} % blue
\newcommand{\red}[1]{{\color{red} #1}} % red
\newcommand{\green}[1]{{\color{darkgreen} #1}} % green
\newcommand{\defn}[1]{\textbf{\textsf{\color{PineGreen} #1}}} % emphasis of a definition
\usepackage{todonotes}
\newcommand{\nantel}[1]{\todo[color=red!30]{#1 \\ \hfill --- N.}}
\newcommand{\Nantel}[1]{\todo[inline,color=red!30]{#1 \\ \hfill --- N.}}
\newcommand{\vincent}[1]{\todo[color=blue!30]{#1 \\ \hfill --- V.}}
\newcommand{\Vincent}[1]{\todo[inline,color=blue!30]{#1 \\ \hfill --- V.}}

% permutations
\newcommand{\fS}{\mathfrak{S}} % symmetric group
\newcommand{\fR}{\mathfrak{R}} % subset symmetric group


% lattices
\newcommand{\meet}{\wedge} % meet
\newcommand{\join}{\vee} % join
\newcommand{\less}{\vartriangleleft} % smaller WOIP
\newcommand{\lesseq}{\trianglelefteq} % smaller WOIP
\newcommand{\more}{\vartriangleright} % larger WOIP
\newcommand{\contactLess}[1]{\less_{#1}} % smaller contact graph
\newcommand{\contactMore}[1]{\more_{#1}} % larger contact graph
\newcommand{\projDown}{\pi_\downarrow} % Down projection
\newcommand{\projUp}{\pi^\uparrow} % Down projection

% Orientation, Hypergraph and Segments
\newcommand{\Or}{\mathcal O}  % The map S_n --> acyclic orientation corresponding to vertices
\newcommand{\HH}{\mathbb H}  % general hypergraph
\newcommand{\II}{\mathbb I} % segment hypergraph
\newcommand{\cJ}{\mathcal J} % basic irreducibles index set
\newcommand{\cA}{\mathcal A} % special set for covers, A(a) involve only segments with source a

%%%%%%%%%%%%%%%%%%%%%%%%%%%%%%%%%%%%%%
%%%%%%%%%%%%%%%%%%%%%%%%%%%%%%%%%%%%%%
%%%%%%%%%%%%%%%%%%%%%%%%%%%%%%%%%%%%%%

\begin{document}

\begin{abstract}
	Lets be real, not just abstract
\end{abstract}

\vspace*{-.8cm}

\maketitle
\nantel{Todo ``Nantel" avec majuscule sont en ligne avec le texte, "nantel" est comme avant}
\Nantel{Vincent, verifie ton affiliation, grant et addresse}


\tableofcontents

%%%%%%%%%%%%%%%%%%%%%%%%%%%%%%%%%%%%%%
%%%%%%%%%%%%%%%%%%%%%%%%%%%%%%%%%%%%%%
%%%%%%%%%%%%%%%%%%%%%%%%%%%%%%%%%%%%%%

\section{Introduction}
\label{sec:introduction}

\begin{theoremA}\label{thm:latticeI}
For a segment hypergraph $\II$ the poset $P_\II$ is a lattice if and only if $\II$ is closed under intersections.
\end{theoremA}

%%%%%%%%%%%%%%%%%%%%%%%%%%%%%%%%%%%%%%
%%%%%%%%%%%%%%%%%%%%%%%%%%%%%%%%%%%%%%
%%%%%%%%%%%%%%%%%%%%%%%%%%%%%%%%%%%%%%

\section{Hypergraphical Polytopes and their partial orders}
\label{sec:HP}


%%%%%%%%%%%%%%%%%%%%%%%%%%%%%%%%%%%%%%
\subsection{Hypergraphic Polytope} 
\label{subsec:D_H}

Let $[n]=\{1,2,\ldots,n\}$. A \defn{hypergraph} $\HH$ on $[n]$, is a collection of  subsets of $[n]$.
By convention, we will assume that for all $i\in [n]$, we have $\{i\}\in \HH$.
Following~\cite[Def 2.11]{BenBerMac}, given an hypergraph $\HH$, we define the \defn{hypergraphic polytope}
$\Delta_{\HH}$ as the the Minkowski sum
 $$\Delta_{\HH} = \sum_{H\in \HH} \Delta_H\,,$$
where $\Delta_H$ is the simplex given by the convex hull of the points $\{e_i |\, i\in H\}\subset \R^n$.

\begin{example}\label{ex:DH}
For the hypergraph
$\HH=\big\{\{1\},\{2\},\{3\},\{4\},\{3,4\},\{1,2,3\}\big\}$,
we  have
$$\begin{array}{ccc}
 \begin{tikzpicture}[scale=1,baseline=.5cm]
	\node (1) at (0.6,1.0) {$\scriptstyle e_1$};
	\node (2) at (-.2,-.2) {$\scriptstyle e_2$};
	\node (3) at (1.5,-.2) {$\scriptstyle e_3$};
	\draw [fill=blue!40] (0,0) -- (.6,.75) -- (1.2,0) --(0,0) ; 
\end{tikzpicture} \quad &
 \begin{tikzpicture}[scale=1,baseline=.5cm]
	\node at (-.2,0) {$\scriptstyle e_3$};
	\node at (1.2,.5) {$\scriptstyle e_4$};
	\draw [thick,color=red] (0,0) -- (1,.5 ); 
\end{tikzpicture} \quad &
\begin{tikzpicture}[scale=1,baseline=.5cm]
	\draw [fill=gray!10] (0,0) -- (.6,.75)-- (1.6,1.25) -- (2.2,.5) -- (1.2,0) --(0,0) ; 
	\draw [color=gray!10,fill=blue!20] (0,0) -- (.6,.75) -- (1.2,0) --(0,0) ; 
	\draw [dotted,color=red] (0,0)--(1,.5);
	\draw [dotted,color=blue] (1.6,1.25)--(1,.5)--(2.2,.5);
	\draw (0,0) -- (.6,.75)-- (1.6,1.25) -- (2.2,.5) -- (1.2,0) -- (0,0) ; 
	\draw [thick,color=red!80] (1.2,0) -- (2.2,0.5); 
\end{tikzpicture}\\
\blue{{ \Delta}_{123}}& \red{{ \Delta}_{34}} & \Delta_{\HH}=\Delta_1+\Delta_2+\Delta_3+\Delta_4+\blue{{ \Delta}_{123}}+ \red{{ \Delta}_{34}}\\
\end{array}
$$
which is a 3-dimensional polytope sitting in $\R^4$. 
\end{example}

\begin{remark}\label{rem:single} In~\cite{BenBerMac}, the authors always assume that for all $i\in[n]$ we have $\{i\}\not\in\HH$ whereas here we took the opposite convention. 
When quoting result from~\cite{BenBerMac}, the reader will have to be mindful of this difference.
We point out, as in  Example~\ref{ex:DH}, that the $\Delta_i$ are single points and in the definition of $\Delta_{\HH}$ it simply translate the polytope in the direction of $e_i$ but does not affect the  face structure of the polytope.
\end{remark}
%%%%%%%%%%%%%%%%%%%%%%%%%%%%%%%%%%%%%%
\subsection{Hypergraphic Posets} 
\label{subsec:P_H}
One of the main result of~\cite[Thm 3.18]{BenBerMac} is to identify the faces of $\Delta_\HH$ with the acyclic orientations of certain quotient of $\HH$.  A useful tool,~\cite[Lem 2.9]{BenBerMac}, gives us a surjective map $\Omega$ from the set composition (ordered set partitions) of $[n]$ to the set of faces of $\Delta_\HH$. 
 In the present work, we will be interested in the skeleton (faces of dimension 0 and 1)
of the polytope as an oriented graph $P_{\HH}$. It will turn out that $P_{\HH}$ is a poset (partial order).  The map $\Omega$ identifies the vertices (0-dim faces) $V_{\HH}$  of $\Delta_{\HH}$ as the image of the set compositions of $[n]$ with $n$ parts. In turn, the set composition with $n$ parts correspond to $\fS_n$ the permutations of the set $[n]$. This gives us a map $\Or \colon \fS_n \to V_{\HH}$.
More explicitly, given $\pi\in\fS_n$ we construct an acyclic orientation $\Or_\pi$ of $\HH$ as follow. For any $H\in \HH$ such that $|H|>1$ we orient $H$  with the source $i$  the leftmost element of $H$ in $\pi$. That is
\begin{equation}\label{eq:orientation}
	\Or_\pi = \Big\{\big(i,H\big)\ \big| \ H\in \HH, |H|>1,  i=\pi\big(\min\{\pi(j): \pi(j)\in H\}\big)\Big\}\,.
\end{equation}
We immediately obtain the following lemma.

\begin{lemma}\label{lem:Hvertices}
For any hypergraph $\HH$, the set of vertices of $\Delta_{\HH}$ is given by $V_{\HH}=\{\Or_\pi: \pi\in \fS_n\}$.
\end{lemma}

We now turn our attention to the edges (1-dim faces) $E_{\HH}$ of $\Delta_{\HH}$. The map $\Omega$ identifies the set $E_{\HH}$ as certain non-trivial image of the set compositions of $[n]$ with $n-1$ parts.
These corresponds  exactly to the cover relation $\sigma=\cdots ab \cdots < \cdots ba\cdots =\pi$ of the (right) weak order on $\fS_n$, where $a<b$. 
Hence given a cover $\sigma < \pi$ in the weak  order, two cases may happen. On one hand it is possible that $\Or_\sigma=\Or_\pi$, In this case, it is a vertex and not an edge. On the other hand, if we get 
 $\Or_\sigma\ne \Or_\pi$, then it is an (oriented) edge $\Or_\sigma\to\Or_\pi$ of $\Delta_{\HH}$. We thus have the following lemma.

\begin{lemma}\label{lem:Hedges}
For any hypergraph $\HH$, the set of (oriented) edges of $\Delta_{\HH}$ is given by 
 $$E_{\HH}=\{(\Or_\sigma,\Or_\pi): \sigma<\pi \text{ is a cover}, \Or_\sigma\ne\Or_\pi\}\,.$$
\end{lemma}

\begin{proposition}\label{prop:PHisOrder}
The oriented graph $P_{\HH}$ given by the transitive closure of $(V_{\HH},E_{\HH})$ is acyclic, hence it defines a poset on $V_{\HH}$.
\end{proposition}

\begin{proof} It suffice to remark that any cycle on $P_{\HH}$ would induce a cycle on the $\fS_n$-weak order, a contradiction.
\end{proof}

The previous proposition allows us to define the \defn{hypergraphic poset} $P_{\HH}$ of any hypergraph $\HH$.  
We can summarize the previous result with the following proposition

\begin{proposition}\label{prop:WeakToP}
With the weak order on $\fS_n$, the map $\Or \colon \fS_n \to P_{\HH}$ is order preserving.
\end{proposition}



\begin{remark}\label{rem:EdgeNotCover}
It is important to notice that the edges in $E_{\HH}$ are not necessarily covers of the relation in $P_{\HH}$. The simplest example is given by $\HH=\big\{\{1\},\{2\},\{3\},\{1,2,3\}\big\}$.
The vertices $V_{\HH}$ of the polytope $\Delta_{\HH}$ are given by
	$$\Big\{ \Or_{123}=\Or_{132}, \Or_{213}=\Or_{231}, \Or_{312}=\Or_{321}\Big\}= \Big\{ \big\{(1,\{1,2,3\})\big\},\big\{(2,\{1,2,3\})\big\},\big\{(3,\{1,2,3\})\big\}\Big\}$$
and the edges
	$$ E_{\HH}=\big\{ (\Or_{123},\Or_{213}),(\Or_{132},\Or_{312}),(\Or_{231}, \Or_{312})\big\}\,.$$
in picture, that gives
$$\Delta_\HH=\Delta_1+\Delta_2+\Delta_3+\Delta_{123}=
\begin{tikzpicture}[scale=.7,baseline=.0cm]
	\node (a) at (0,-1.2) {$\scriptscriptstyle \{(1,\{1,2,3\})\}$};
	\node (b) at (3,0) {$\scriptscriptstyle \{(2,\{1,2,3\})\}$};
	\node (c) at (0,1.2) {$\scriptscriptstyle \{(3,\{1,2,3\})\}$};
	\draw [color=blue!40,thick,fill=blue!20] (0,-1)--(0,1)--(1.632,0)--(0,-1) ; 
	\draw [color=red,thick,->] (-.1,-1)--(-.1,1); 
	\draw [color=red,thick,->] (1.832,0)--(.2,1); 
	\draw [color=red,thick,->] (.2,-1)--(1.832,0); 
\end{tikzpicture}
$$
and clearly the edge $(\Or_{132},\Or_{312})$ is not a cover. 
\end{remark}

%%%%%%%%%%%%%%%%%%%%%%%%%%%%%%%%%%%%%%
\subsection{Sources and preimage of $\Or$} 
\label{subsec:notation}

We have seen that the map $\Or \colon \fS_n \to V_{\HH}$ is surjective.
We now set some notation in this context. Let $\HH$ be  an hypergraph on $[n]$. Given a permutation $\pi\in\fS_n$ and $H\in \HH$,
we say that 
\begin{equation}\label{eq:source}
	S(H,\Or_\pi)=\pi\big( \min\{j:\pi(j)\in H\}\big)
\end{equation}
is the \defn{source} of $H$ for the orientation $\Or_\pi$. 

Given an acyclic orientation $A\in P_\HH$,  we can characterize the preimage 
	$$ \Or^{-1}(A)=\{ \pi : \Or_\pi=A\}$$
as follow. Let $\less_A$ be the order on $[n]$ defined by the transitive closure of the union of 
the order $\less_{(i,H)}=\big\{ i< j : j\in H\ssm\{i\}\big\}$ for each $(i,H)\in A$ . That is
 	$$\less_A =  \bigcup_{(i,H)\in A} 
	\begin{tikzpicture}[scale=1,baseline=.0cm]
	\node at (0,-.45) {$\scriptstyle i$};
	\node at (0,.6) {$\scriptstyle j\in H\ssm \{i\}$};
	\node at (.2,.35) {$\ldots$};
	\draw [thick] (0,-.3)--(-.5,.4); 
	\draw [thick] (0,-.3)--(-.3,.4); 
	\draw [thick] (0,-.3)--(-.1,.4); 
	\draw [thick] (0,-.3)--(.5,.4); 
	\end{tikzpicture}
	$$
This is a well defined order since $A$ is acyclic. The following is a straightforward lemma left to the reader to prove.
\begin{lemma}\label{lem:prepi}   $ \Or^{-1}(A) =\big\{ \pi : \pi \text{ is a linear extension of }  \less_A)\big\}$.
\end{lemma}


In General, given a hypergraph $\HH$ we would be interested to understand what are the cover of $P_{\HH}$? 
When is $P_{\HH}$ a lattice? distributive lattice? a semi-lattice? (semi-)lattice quotient of the $fS_n$-weak lattice?
Can we have a better description of $\Or{-1}$? and more.

In the present work we focus our attention to a subfamily of Hypergraph, namely we assume that all $H\in \HH$ are segment.

%%%%%%%%%%%%%%%%%%%%%%%%%%%%%%%%%%%%%%
%%%%%%%%%%%%%%%%%%%%%%%%%%%%%%%%%%%%%%
%%%%%%%%%%%%%%%%%%%%%%%%%%%%%%%%%%%%%%

\section{Segment Hypergraphic Polytopes and their partial orders}
\label{sec:IHP}

In this paper, we  assume that $\HH=\II$ is an hypergraph where all $I\in \II$ are segment of the form $I=[a,b]=\{a,a+1,a+2,\ldots,b\}$.
We call such hypergraph a \defn{segment hypergprah}. 

\begin{example}\label{ex:seghyp}
 $\II=\{\{1\},\{2\},\{3\},\{4\},[1,3],[2,3],[2,4],[1,4]\}$ is a segment hypergraph and $\Or_{4132} =\big\{ (4,\{1,2,3,4\}), (1,\{1,2,3\}),(3,\{2,3\}),(4,\{2,3,4\})\big\}$  is an example of 
 acyclic orientation in $P_\II$.
We will represent segment hypergraphs and their orientations graphically as follow:
 	$$\II =  
	\begin{tikzpicture}[scale=1,baseline=.0cm]
	\foreach \x in {1,...,4}
		\node (\x) at (\x*.5,-.4) [inner sep = -1pt] {$\scriptstyle \x$};
	\draw [thick,{Bar[width=3pt]}-{Bar[width=3pt]}] (.5,-.2)--(2,-.2); 
	\draw [thick,{Bar[width=3pt]}-{Bar[width=3pt]}] (.5,0)--(1.5,0); 
	\draw [thick,{Bar[width=3pt]}-{Bar[width=3pt]}] (1,.2)--(1.5,.2); 
	\draw [thick,{Bar[width=3pt]}-{Bar[width=3pt]}] (1,.4)--(2,.4); 
	\end{tikzpicture}
	\qquad\qquad
	\Or_{4132}  =  
	\begin{tikzpicture}[scale=1,baseline=.0cm]
	\foreach \x in {1,...,4}
		\node (\x) at (\x*.5,-.4) [inner sep = -1pt] {$\scriptstyle \x$};
	\draw [thick,{Bar[width=3pt]}-{Bar[width=3pt]}] (.5,-.2)--(2,-.2);  \node at (2,-.2) {$\bullet$};
	\draw [thick,{Bar[width=3pt]}-{Bar[width=3pt]}] (.5,0)--(1.5,0);   \node at (.5,0) {$\bullet$};
	\draw [thick,{Bar[width=3pt]}-{Bar[width=3pt]}] (1,.2)--(1.5,.2);   \node at (1.5,.2) {$\bullet$};
	\draw [thick,{Bar[width=3pt]}-{Bar[width=3pt]}] (1,.4)--(2,.4);   \node at (2,.4) {$\bullet$};
	\end{tikzpicture}
	$$
and we omit to draw the singleton $\{i\}$ for $1\le i\le n$.
\end{example}



%%%%%%%%%%%%%%%%%%%%%%%%%%%%%%%%%%%%%%
\subsection{Acyclic orientation of segment hypergraph}  
\label{subsec:sourceinc}

Let $\II$ be a segment hypergraph.
In this section we give a simple characterization of the orientation of $\II$ that are acyclic.
Given any orientation (not necessarily acyclic) $O=\{(a_I,I)|\, a_I\in I\in \II,  |I|>1\}$ of $\II$.
If the orientation $O$ has a cycle, then we can find $(a_1,I_{1}), (a_2,I_{2}),\ldots (a_m,I_{m})$ in $O$, such that
$a_{s+1}\in I_s$ for $1\le s<m$, and $a_1\in I_m$. 
$$
	\begin{tikzpicture}[scale=1,baseline=.0cm]
	\node at (-.5,1) {$\scriptscriptstyle I_m$}; \draw [thick,{Bar[width=3pt]}-] (.3,1)--(1.2,1);   \node at (1.3,1) {$\scriptstyle \cdot$};\node at (1.4,1) {$\scriptstyle \cdot$};\node at (1.5,1) {$\scriptstyle \cdot$};
	\node at (1,0.9) {$\scriptstyle \cdot$}; \node at (1.1,0.8) {$\scriptstyle \cdot$}; \node at (1.2,0.7) {$\scriptstyle \cdot$}; 
	\node at (-.5,0.6) {$\scriptscriptstyle I_4$}; \draw [thick,{Bar[width=3pt]}-{Bar[width=3pt]}] (2,.6)--(3,.6);  \node at (2.5,.6) {$\bullet$};
	\node at (-.5,0.4) {$\scriptscriptstyle I_3$}; \draw [thick,{Bar[width=3pt]}-{Bar[width=3pt]}] (2.5,.4)--(4.5,.4);   \node at (3.5,.4) {$\bullet$};
	\node at (-.5,0.2) {$\scriptscriptstyle I_2$}; \draw [thick,{Bar[width=3pt]}-{Bar[width=3pt]}] (1,.2)--(4,.2);   \node at (1.5,.2) {$\bullet$};
	\node at (-.5,0) {$\scriptscriptstyle I_1$}; \draw [thick,{Bar[width=3pt]}-{Bar[width=3pt]}] (0,0)--(1.5,0);   \node at (.5,0) {$\bullet$};
	\end{tikzpicture}
	$$
We may assume that $m$ is the first time $a_1\in I_m$ and  $a_s\not\in I_m$ for $1<s<m$.
By symmetry, let us assume $a_1<a_2$. Since $a_2\in I_1\ssm I_m$, then $\max(I_m)<\max(I_1)$. If we suppose that $a_m\not\in I_1$, then $a_m<a_1<a_2$.
We thus have  $a_1\in\bigcup_{2\le s<m} I_s$ since $a_2\in I_2$ and $a_m\in I_{m-1}$. We thus have $a_1\in I_s$ for some $s<m$, a contradiction to our choice of $m$.
Therefore, we have $m=2$.

\begin{proposition}\label{prop:acyclic}
 For a segment hypergraph $\II$, an orientation $O$ of $\II$ is acyclic if and only if there is no $(i,I),(j,J)\in O$ such that $i\in J$ and $j\in I$.
 Graphically, there is no patterns
 $$
	\begin{tikzpicture}[scale=1,baseline=.0cm]
	\node at (-.5,0.2) {$\scriptscriptstyle J$}; \draw [thick,{Bar[width=3pt]}-] (0,.2)--(1.8,.2);   \node at (1.5,.2) {$\bullet$};   \node at (1.9,.2) {$\scriptstyle \cdot$};\node at (2,.2) {$\scriptstyle \cdot$};\node at (2.1,.2) {$\scriptstyle \cdot$};
	\node at (-.5,0) {$\scriptscriptstyle I$};  \node at (-.1,0) {$\scriptstyle \cdot$};\node at (0,0) {$\scriptstyle \cdot$};\node at (.1,0) {$\scriptstyle \cdot$}; \draw [thick,-{Bar[width=3pt]}] (.2,0)--(2,0);   \node at (.5,0) {$\bullet$};
	\node at (.5,-.2) {$\scriptscriptstyle i$}; \node at (1.5,-.2) {$\scriptscriptstyle j$}; 
	\end{tikzpicture}
	$$
\end{proposition}

\begin{proof}
If the given orientation contain such pattern, clearly it is not acyclic. The converse was argued just before the proposition.
\end{proof}

\begin{remark} \label{rem:separation}
Given an acyclic orientation $A\in P_\II$,  the condition in Proposition~\ref{prop:acyclic} implies that the order $\less_A$ is \defn{separating}. That is, for any $I\in \II$ such that $|I|>1$,
and for any $i<S(I,A)<j$, we have that $i$ and $j$ are incomparable in $\less_A$. 
\end{remark}

we add a  technical lemmas about  the linear extension of separating order on $[n]$ that will be useful to us. In general, we say an order $\less$ on $[n]$ is \defn{separating} if 
for any $a\less b$ and any $i<a<j$, we have that $i$ and $j$ are incomparable in $\less$.

\begin{lemma} \label{lem:XaWbY}
Given a separating order $\less$ on $[n]$ and  $a,b\in [n]$ such that $a<b$ and $a\less b$. We can find $\pi= XaWbY$ a linear extension of $\less$ such that
$a<w$ and $a\less w\less b$ for all $w\in W$. Moreover, we could choose $W$ such that $a<w<b$ and $a\less w\less b$ for all $w\in W$, if and only if  there is no $b<c$ such that $a\less c\less b$.
\end{lemma}

\begin{proof} We first find a linear extension $X_0aX_1$ of the restriction of the order $\less$ on $[1,a]\subseteq [n]$.
Similarly, we find a linear extension $Y_0aY_1$ of the restriction of $\less$ on $[a,n]$. Since all elements of $X_0,X_1$ are separated by $a$ to all elements of $Y_0,Y_1$,
then every element of $X_0,X_1$ is incomparable in $\less$ to every element of $Y_0,Y_1$. Therefor $X_0Y_0aY_1X_1$ is a linear extension of $\less$.
In choosing linear extensions we can always assume that all incomparable elements  are placed before $a$, thus in $Y_0$ and $X_0$. 
Thus for all $y\in Y_1$, we have $a<y$ and $a\less y$ and in particular $b\in Y_1$. We have $Y_1=WbY_2$ and $W$ is as desired by choosing to put all  $c\in Y_1$ incomparable to $b$ in $Y_2$.
Moreovere, if there is no $b<c\in Y_1$ such that $c\less b$, then  $a<w<b$ and $a\less w\less b$ for all $w\in W$. The converse holds since if $b<c\in W$ for all $W$ implies $c\less b$.
\end{proof}

%%%%%%%%%%%%%%%%%%%%%%%%%%%%%%%%%%%%%%
\subsection{Preimage $\Or^{-1}$ for segment hypergraph} 
\label{subsec:preimageI}
One striking property for segment hypergraph is that  the sets  $\Or^{-1}(A)$ is an interval in the weak order.
To describe this we first need to recall a classic result about linear extension of orders on $[n]$.

\begin{proposition}[{\cite[Thm.~6.8]{BjornerWachs}}]
\label{prop:WOIP}
The set of linear extensions of a poset~$\less$ on~$[n]$ forms an interval~$I$ of the weak order if and only if for every~$i < j < k$,
\[
i \less k \implies i \less j \text{ or } j \less k
\qquad\text{and}\qquad
i \more k \implies i \more j \text{ or } j \more k.
\]
Moreover, the inversions of~$\min(I)$ are the pairs~$i,j \in [n]$ with $i < j$ and $i \more j$, and the non-inversions of~$\max(I)$ are the pairs~$i,j \in [n]$ with $i < j$ and $i \less j$.
\end{proposition}


\begin{proposition}\label{prop:preimage}
 Given a segment hypergraph $\II$, we have that for any vertex $A\in V_{\II}$, the set $\Or^{-1}(A)$ is an interval 
 in the weak order with minimum avoiding the pattern $231$ and maximum avoiding the pattern $213$.
\end{proposition}

\begin{proof}
From Lemma~\ref{lem:prepi} we have that any $\pi\in \Or^{-1}(A)$  is a linear extension of $\less_A$, we can therefor use Proposition~\ref{prop:WOIP} to prove our claim. 
Let $i<j<k$ be any integers in the natural order of $[n]$. If we have $i\less_{A} k$, then, by definition of $\less_A$, there must be a sequence $(i_1,I_{1}), (i_2,I_{2}),\ldots (i_m,I_{m})$ of oriented edges in $A$ such that
$i=i_1<i_2<\cdots <i_{m}$, for $1\le s<m$ we have $i_{s+1}\in I_s$, and $k\in I_m$. We must therefor have $j\in\bigcup_{1\le s\le m} I_s$ since both $i$ and $k$ are in the union. This implies $i\less_{A} j$. The case $i\more_{A} k$ is similar and  implies
that $j\more_A k$. This shows $\Or^{-1}(A)$ is an interval in the weak order.

For the remaining of our claim, assume $\pi$ is the maximum of the interval of $\Or^{-1}(A)$ and has a $213$ pattern. That is,  we can find $i<j<k$ such that $\{i,k\}$ is a non-inversion of $\pi$ and $\{i,j\}$ is an inversion. Using Proposition~\ref{prop:WOIP} we must have $i\less_A k$ and $i\not\less_A j$, a contradiction to our proof above. The case for the minimum is similar.
\end{proof}



%%%%%%%%%%%%%%%%%%%%%%%%%%%%%%%%%%%%%%
\subsection{A source characterization of $P_\II$}  
\label{subsec:sourceinc}

A necessary and sufficient  condition of the order $P_\II$ is that sources must increase for each $I\in \II$.

\begin{proposition}
\label{prop:sourceorder}
Let $\II$ be a segment hypergraph. For any $A,B\in P_\II$ we have
$$ A\le B\quad  \text{ if and only if }\quad  S(I,A)\le S(I,B) \text{ for all } I\in \II .$$
\end{proposition}
\begin{proof} Let $\pi$ be any permutation such that $A=\Or_\pi$. From Proposition~\ref{prop:WeakToP}, the order $P_\II$ is generated by  weak cover relations on $\fS$,
so let $\pi<\pi'$ be a cover. That is $\pi'=\pi(i,i+1)$ where $a=\pi(i)<\pi(i+1)=b$. Fix $I\in \II$. If both $a,b\in I$ and $a=S(I,A)$, then $b=S(I, \Or_{\pi'})$.
In all other cases $S(I,A)=S(I,\Or_{\pi'})$. The forward implication $\big(A\le B \implies S(I,A)\le S(I,B)\  \forall( I\in \II)\big)$ then follows by induction on the distance from $A$ to $B$.

For the converse, assume $S(I,A)\le S(I,B)$  for all $ I\in \II$. Let 
	$$\delta_{A,B}=\big\{(a,b)\  |\  a\not= b, a=S(K,A), b=S(K,B)\text{ for some } K\in \II\big\}.$$
If $\delta_{A,B}=\emptyset$, then we have $A=B$ and the result holds. Otherwise, choose $(a,b)\in \delta_{A,B}$ such that $b$ is maximal, and for this $b$, let $a$ is maximal.
We choose $\pi=XaWbY$ as in Lemma~\ref{lem:XaWbY} for $\less_A$ so that $A=\Or_\pi$. For $\pi'=(a,b)\pi$, we claim the orientation $A'=\Or_{\pi'}$ satisfies $A<A'$ and  that for all $I\in \II$ we have $S(I,A')\le S(I,B)$.

To show the claim, first assume there is $c\in W$ such that $c>b$ and $c\less_A b$. Hence there is $\{(c_s,I_s)\}\subseteq A$ such that (using separation property) $c=c_1>c_2>\cdots>c_{m}>b$ 
with $c_{s+1}\in I_s$ for $1\le s<m$ and $b\in I_m$. By choice of $b$,  we must have $\{(c_s,I_s)\}\subseteq B$ since all $c_s>b$. This gives $c\less_B b$. 
Now we use $a\less_A c$ (true for any $c\in W$) which implies a set
$\{(a_r,I_r)\}\subseteq A$ such that $a=a_1<a_2<\cdots<a_{p}<c$ 
with $a_{r+1}\in I_r$ for $1\le r<p$ and $c\in I_{p}$. Since $a<b<c$ we have that $b\in \bigcup_{1\le r\le p} I_r$, so $b\in I_{r_0}$ for some $1\le r_0\le p$. 
Choose $r_0$ to be maximal with this property. Hence for all $r_0<r\le p$ we have  $b< \min(I_r)$ and $a'_r=a_r$ by maximality of $b$. By choice of $b$, 
there is a $K\in\II$ such that $(b,K)\in B$. We also have $(a'_{r_0},I_{r_0})\in B$ where 
 $$\min(I_{r_0+1})\le a_{r_0}\le a'_{r_0} \le \max(I_{r_0}) < a_{r_0+1} \le \max(I_{r_0+1}),$$
 hence $a'_{r_0} \in I_{r_0+1}$.
The set $\{(b,K),(a'_{r_0},I_{r_0})\} \cup \{(a_r,I_r)\}_{r_0<r\le p} \subseteq B$ shows that $b\less_B c$ a contradiction to the acyclicity of $B$. 
Therefore, there is no $c>b$ in $W$. All the inversion of $\pi=XaWbY$ are inversion of $\pi'=XbWaY$ since $a<w<b$ for all $w\in W$, which gives $A<A'$.

For the second part of the claim consider any $I\in \II$. if $\max(I)<a$ or $b<\min(I)$ or $a<\min(I)<\max(I)<b$, then $S(I,A')=S(I,A)\le S(I,B)$. 

If $a<\min(I)\le b\le \max(I)$, then either $S(I,A')=S(I,A)\le S(I,B)$
or $a<c=S(I,A)<S(I,A')=b$. In the second case, we must have that $S(I,B)=b$, otherwise if $S(I,B)=c'<b$ then $(c',I)$ and $(b,K)$ would be a cycle 
(recal that there is a $(b,K)\in B$ such that $a\in K$, $a<c\le c'<b$, therefore $c\in K$). The choice of $b$ eliminate the possibility that $S(I,B)=c'>b$, hence $S(I,A')=b=S(I,B)$ in this case.

If $\min(I)\le a<b\le \max(I)$, then if $a\le S(I,A)\le b$ we must have  $S(I,B)=b$ as it cannot be greater by choice of $b$ and it cannot be smaller, creating a cycle with $(a,K)$. 
if $S(I,A)<a$, then our choice of $\pi$ implies that $c$ is before $a$, therefor $S(I,A')=A(I,A)\le S(I,B)$. If $S(I,A)=c>b$, then we have shown above that  there is no $c>b$ in $W$ for $\pi$.
Hence, $c$ is before $a$ in $\pi$ and we have again $S(I,A')=A(I,A)\le S(I,B)$.

Finally, if $\min(I)\le a\le \max(I)<b$, then we cannot have $S(I,A)=c>a$ for this would create a cycle with $(a,K)\in A$. If $S(I,A)=c<a$, then as above $c$ is before $a$ in $\pi$ and we have 
$S(I,A')=S(I,A)\le S(I,B)$. If $S(I,A)=a$, then $a \le S(I,A')=c<b$. Assume that $a<c<b$. Having $S(I,A')=c$ means that $c\in W$ in $\pi$, which in turn implies that $a\less_A c \less_A b$.
We use $c\less_A b$ as before to exhibit $\{(c_s,I_s)\}\subseteq A$ with $c=c_1<c_2<\cdots <c_m<b$ where $c_{s+1}\in I_s$ and $b\in I_m$.
The segment $I_m$ cannot contain $a<c\le c_m$ for otherwise we would get a cycle $a\less_A c_m \less_A a$. In $B$, we must have $S(I_m,B)=d=b$ since $c_m<b$ and the choice of $b$ prevent $d>b$,
and $d<b$ is not allowed by $(b,K)$ and $a<\min(I_m)\le b$ (it would create a cycle). Now $a<c_m=S(I_m,A)<S(I_m,B)=b$ contradict the choice of $a$. Hence we can only have $S(I,A)=c<a$ which 
conclude the end of our claim. 

The proposition then follow by induction on the size of $\delta_{A,B}$ since $\delta_{A',B}\subsetneq \delta_{A,B}$.



\Nantel{The argument of Vincent in the note is incomplete. What guaranties that flipping $a$ and $b$ gives comparable elements?
I priori, even if we show it is a valid orientation, it may give us something incomparable? We have to show it is obtained by a  weak order relations as done above.}
\end{proof}

%%%%%%%%%%%%%%%%%%%%%%%%%%%%%%%%%%%%%%
\subsection{Flips and covers in $P_\II$}  
\label{subsec:cover}

In Remark~\ref{rem:EdgeNotCover} we saw that a cover in the weak order does not always give us a cover  in the order $P_\II$. Yet, it is a necessary condition, hence if
$A<B$ is a cover  $P_\II$ we must be able to find a cover in the weak order $\pi_A<\pi_B$  such that $A=\Or_{\pi_A}$, $B=\Or_{\pi_B}$. The permutations $\pi_A<\pi_B$ 
must satisfy $\pi_B=\pi_A(i,i+1)$ for some $1\le i<n$, where $a=\pi_A(i)<\pi_A(i+1)=b$.
This will affect only the segment $I\in \II$ such that $[a,b]\subseteq I$ and such that $S(I,A)=a$. Indeed
\begin{equation}\label{eq:PossibleCover}
 	S(I,B)=\begin{cases}
		b& \text{if $S(I,A) = a$ and $[a,b]\subseteq I$},\\
		S(I,A)& \text{otherwise.}
	\end{cases}
\end{equation}

This suggests the following definition. A \defn{flip} in $A$ is two integer $a<b$ such that the orientation $B=(a,b)A$ defined by Equation~\eqref{eq:PossibleCover}
is acyclic and $B\ne A$. Using Proposition~\ref{prop:sourceorder}, we have that if $B=(a,b)A$ is a flip, then $A<B$. 


Our first task is to characterize flips. Let
 $$\cA(a)=\bigcup_{(a,I)\in A} \cA_{(a,I)}\,,\qquad\text{where}\qquad
 \cA_{(a,I)}= I \ssm  \bigcup_{(c,J)\in A \atop a\ne c\in I} (J\ssm \{c\})\,.$$
By construction, it is clear that if $b\in (I\ssm \cA_{(a,I)})$, then changing the source of $I$ from $a$ to $b$ would create a cycle with some $(c,J)$ and not be acyclic.

\begin{lemma}\label{lem:isaflip}
$B=(a,b)A$ is a flip if and only if $b\in \cA(a).$
\end{lemma}

\begin{proof}
If $b\not\in \cA (a)$, then either $b\not\in I$ for all $(a,I)\in A$, or there are $(a,I),(c,J)\in A$ such that $c\in I$ and  $b\in I\cap J$. In the first case $B=(a,b)A=A$ is not a flip, and in the second cases $(b,I),(c,J)$ is a cycle so 
$B=(a,b)A$ is not acyclic, hence not a flip.  For the converse, if $a<b\in \cA (a)$, then there is at least one $(a,I)\in A$ such that $b\in I$. The orientation $B=(a,b)A\ne A$ since Equation~\eqref{eq:PossibleCover} change at least one source.
For all of these change,  the definition of  $\cA_{(a,I)}$ guaranties there is no $(b,I)(c,J)\in B$ such that $b\ne c$ and $b,c\in I\cap J$. Hence, Proposition~\ref{prop:acyclic} gives us that $B$ is acyclic and thus $B=(a,b)A$ is a flip.
\end{proof}

As described above, if $A<B$ is a cover, then we have $\pi_A<\pi_B$ is a cover which gives us $B=(a,b)A$ is a  flip.
To understand the flips that are not covers we need to look at the set
 $$\cA(a,b)= \Big\{ i\in \cA (a) \,\big|\  a<i<b\  \text { and }\  i\not\in \!\!\!\bigcup_{(a,J)\in A \atop \max(J)<b} J \Big\}.$$
 
\begin{proposition} \label{prop:segcover}
$A<B$  is a cover if and only if it is a flip for some $a<b$ and $ \cA(a,b)=\emptyset.$
\end{proposition}

\begin{proof}
If $A<B$  is a cover, then it is a flip $B=(a,b)A$ for some $a<b$.
Suppose $i\in  \cA(a,b)$, then $i\in \cA(a)$ and $B'=(a,i)A$ is a flip. This gives $A<B'$. By definition of $\cA(a,b)$, for any $(a,I)\in A$ such that $i\in I$, we must have $b\in I$ as well.
Hence $S(I,B')=i<b=S(I,B)$ for all such $(a,I)$. In all other cases we have $S(I,B')=S(I,A)\le S(I,B)$. Hence $B'<B$ contradicting that $A<B$ is a cover.

Conversely, suppose $ \cA(a,b)=\emptyset$, that $B=(a,b)A$ is a flip but  is not a cover $A<B$.
We can thus find a cover $A<B'$ such that $B'<B$.
Using Proposition~\ref{prop:sourceorder}, we have that $A<B$ have the same source on all $[a,b]\not\subseteq I\in \II$ or $S(I,A)\ne a$. 
Hence the cover $A<B'$ can only be achieved with $B'=(a,i)A$ for some $a<i<b$. Lemma~\ref{lem:isaflip} implies that $i\in \cA(a)$.
Since $i\not\in \cA(a,b)$, then there is $(a,J)\in A$ such that $\max(J)<b$ and $i\in J$. For this particular $J$, we have $S(J,B) = a < i = S(J,B')$
wich implies $B'\not < B$, a contradiction.
\end{proof}

%%%%%%%%%%%%%%%%%%%%%%%%%%%%%%%%%%%%%%
%%%%%%%%%%%%%%%%%%%%%%%%%%%%%%%%%%%%%%
%%%%%%%%%%%%%%%%%%%%%%%%%%%%%%%%%%%%%%

\section{Characterization of  $P_\II$ that are lattices}
\label{sec:LatticePI}

In this section, we  prove  Theorem~\ref{thm:latticeI} characterizing the segment hypergraphs $\II$ for which $P_\II$ are lattices.

%%%%%%%%%%%%%%%%%%%%%%%%%%%%%%%%%%%%%%
\subsection{If $P_\II$ is a lattice, then $\II$ is closed under intersections}  
\label{subsec:latticeI}

We are now ready to show the forward implication of Theorem~\ref{thm:latticeI}.

\begin{proposition}
	For a segment hypergraph $\II$, if the poset $P_\II$ is a lattice, then $\II$ is closed under intersections.
\end{proposition}

\begin{proof} By contradiction, assume we have $I,J\in \II$ such that $\emptyset \not = I\cap J\not\in \II$. Let $a<b<c<d$ be such that $a=b-1$, $b=\min(I\cap J)$, $c=\max(I\cap J)$, and $d=c+1$.
Remark that $b\ne c$ since $\II$ contains all singletons. By symmetry, we assume that $a\in I\ssm J$ and $d\in J\ssm I$. Let $X$ be the element of $[n]\ssm\{a,b,c,d\}$ written in increasing order.
We now construct four permutations 
$$ \pi_A=bacdX,\qquad \pi_B=acdbX,\qquad \pi_C=dbacX \qquad\text{and}\qquad \pi_D=cdbaX,$$
and consider the four distinct orientations $A=\Or_{\pi_A}$, $B=\Or_{\pi_B}$, $C=\Or_{\pi_C}$ and $D=\Or_{\pi_D}$. We display bellow the four orientations highlighting only the segment $I$ and $J$.
$$
\begin{array}{lcr}
	C =  
	\begin{tikzpicture}[scale=1,baseline=.0cm]
	\node at (-.3,.3) {$\scriptstyle I$}; \draw [thick,{Bar[width=3pt]}-{Bar[width=3pt]}] (0,.3)--(1.2,.3);   \node at (.3,.3) {$\bullet$};
	\node at (-.3,0) {$\scriptstyle J$};  \draw [thick,{Bar[width=3pt]}-{Bar[width=3pt]}] (.3,0)--(1.5,0);   \node at (1.5,0) {$\bullet$};
	\node at  (0,-.3) {$\scriptstyle a$};
	\node at  (.3,-.3) {$\scriptstyle b$};
	\node at  (1.2,-.3) {$\scriptstyle c$};
	\node at  (1.5,-.3) {$\scriptstyle d$};
	\end{tikzpicture} 
	&\qquad\quad&
	\begin{tikzpicture}[scale=1,baseline=.0cm]
	\node at (2,.3) {$\scriptstyle I$}; \draw [thick,{Bar[width=3pt]}-{Bar[width=3pt]}] (0,.3)--(1.2,.3);   \node at (1.2,.3) {$\bullet$};
	\node at (2,0) {$\scriptstyle J$};  \draw [thick,{Bar[width=3pt]}-{Bar[width=3pt]}] (.3,0)--(1.5,0);   \node at (1.2,0) {$\bullet$};
	\node at  (0,-.3) {$\scriptstyle a$};
	\node at  (.3,-.3) {$\scriptstyle b$};
	\node at  (1.2,-.3) {$\scriptstyle c$};
	\node at  (1.5,-.3) {$\scriptstyle d$};
	\end{tikzpicture} 
	=D 
	\\ \\ \\
	A =  
	\begin{tikzpicture}[scale=1,baseline=.0cm]
	\node at (-.3,.3) {$\scriptstyle I$}; \draw [thick,{Bar[width=3pt]}-{Bar[width=3pt]}] (0,.3)--(1.2,.3);   \node at (.3,.3) {$\bullet$};
	\node at (-.3,0) {$\scriptstyle J$};  \draw [thick,{Bar[width=3pt]}-{Bar[width=3pt]}] (.3,0)--(1.5,0);   \node at (.3,0) {$\bullet$};
	\node at  (0,-.3) {$\scriptstyle a$};
	\node at  (.3,-.3) {$\scriptstyle b$};
	\node at  (1.2,-.3) {$\scriptstyle c$};
	\node at  (1.5,-.3) {$\scriptstyle d$};
	\end{tikzpicture} 
	&\qquad\quad&
	\begin{tikzpicture}[scale=1,baseline=.0cm]
	\node at (2,.3) {$\scriptstyle I$}; \draw [thick,{Bar[width=3pt]}-{Bar[width=3pt]}] (0,.3)--(1.2,.3);   \node at (0,.3) {$\bullet$};
	\node at (2,0) {$\scriptstyle J$};  \draw [thick,{Bar[width=3pt]}-{Bar[width=3pt]}] (.3,0)--(1.5,0);   \node at (1.2,0) {$\bullet$};
	\node at  (0,-.3) {$\scriptstyle a$};
	\node at  (.3,-.3) {$\scriptstyle b$};
	\node at  (1.2,-.3) {$\scriptstyle c$};
	\node at  (1.5,-.3) {$\scriptstyle d$};
	\end{tikzpicture} 
	=B
\end{array}
$$
We have that $\pi_A<\pi_C$, $\pi_A<\pi_D$ and $\pi_B<\pi_D$ in the weak order. Proposition~\ref{prop:WeakToP} implies that $A<C$, $A<D$ and $B<D$ in the $P_\II$ order.
We claim that $B<C$  but this does not follow directly from the weak order since $\pi_B\not<\pi_C$. To show our claim, consider $\pi_E=adcbX$ and $\pi_F=adbcX$.
For any $K\in \II$ such that $a,d\not\in K$, we cannot have both $c,d\in K$, since this would imply that $K=[c,d]=I\cap J\not\in \II$, a contradiction.
This shows that $E=\Or_{\pi_E}=\Or_{\pi_F}=F$. Now we have $\pi_B<\pi_E$ and $\pi_F<\pi_C$ which gives $B<E=F<C$.
 
If the poset $P_\II$ is a lattice, then we can find a unique join $M= A \join B$. By definition $A\le M$ and $B\le M$. By property of join we also have $M\le C$ and $M\le D$ since both $C$ and $D$ are  greater than $A$ and $B$.
Let $\pi_M$ be any permutation such that $M=\Or_{\pi_M}$ and let  
   $$m=\pi\big(\min\{i:\pi_M(i)\in I\cup J\}\big).$$
If $m<b$, then $(m,I)\in M$ and $(b,I)\in A$ and the easy forward implication of Proposition~\ref{prop:sourceorder} implies that $A\not\le M$.
By a similar argument: $\big(b\le m<c \implies B\not\le M\big)$,  $\big(b< m\le c \implies M\not\le C\big)$, and $\big(c<m \implies M\not\le D\big)$. This is a contradiction to the existence of $m$.
\end{proof}



%%%%%%%%%%%%%%%%%%%%%%%%%%%%%%%%%%%%%%
\subsection{Properties of $\II$ is closed under intersections}  
\label{subsec:IntClosedI}

For the converse of Theorem~\ref{thm:latticeI}, we need to investigate the properties of segment hypergraph that are closed under intersections.
Recall from Proposition~\ref{prop:preimage} that given any $A\in P_\II$ the set $\Or^{-1}(A)$ is an interval $[\pi^0_A,\pi^1_A]$ in the weak order.
In the following we will keep this convention that $\pi^0_A$ denote the minimum of this interval and $\pi^1_A$ the maximum.

\begin{theorem}\label{thm:propertieofintI}
Assume that $\II$ is a segment hypergraph that is closed under intersections. Let $A$ and $B$ be two acyclic orientations of $\II$.
The following are equivalent.
\begin{enumerate}
	\item[{\rm (a)}] $A\le B$ in $P_\II$,
	\item[{\rm (b)}]  For all $I\in\II$, $S(I,A)\le S(I,B)$,
	\item[{\rm (c)}]  $\pi^0_A \le \pi^1_B$ in the weak order,
	\item[{\rm (d)}]  For all $i<j$, \ $j\,\less_Ai \ \ \implies\ \  i \,\not\less_B j$
\end{enumerate}
\end{theorem}

\begin{proof} The equivalence (a)$\iff$(b) follows from Proposition~\ref{prop:sourceorder}. The implication (c)$\implies$(a) follows from  Proposition~\ref{prop:WeakToP}.
For the equivalence (c)$\iff$(d), remark that Proposition~\ref{prop:preimage} combined with Proposition~\ref{prop:WOIP} gives us that the set of inversions of $\pi^0_A$
is $\big\{(i,j)\ |\  i<j, j\,\less_Ai \big\}$ and the set of non-inversions of $\pi^1_B$ is  $\big\{(i,j)\ |\  i<j, i\,\less_B j \big\}$. Hence the inversion of $\pi^1_B$ is  given $\big\{(i,j)\ |\  i<j, i\,\not\less_B j \big\}$.
The equivalence of (c)$\iff$(d) is the characterization of the weak oder in term of inclusion of inversion sets.

Thus we only need to prove (b)$\implies$(d). For a contradiction, assume that $A\le B$ and we have some $i<j$ such that $j\less_Ai$ and  $i \less_B j$. Choose one such pair such that $|j-i|$ is minimal.
Since $j\,\less_A i$ and $i \less_B j$, we can find
$\{(j_s,I_{s})\}\subseteq A$ such that $j=j_1>j_2>\cdots>j_m>i$, each $j_{s+1}\in I_s$ and $i\in I_m$, and 
$\{(i_r,I_{r})\}\subseteq B$ such that $i=i_1<i_2<\cdots<i_p<j$, each $i_{r+1}\in I_r$ and $j\in I_p$. Remark that $i<j_m$ with $j_m\less_A i$ and $i\less_B j_m$, since $j_m\in \bigcup_{1\le r\le p}I_r\supseteq [i,j]$.
Our choice of minimality of $|i-j|$ prevent $m>1$. A similar argument in the other direction shows that we must have $p=1$. 

If we have $i<j$ such that $j\less_Ai$ and  $i \less_B j$, then if we choose $|j-i|$ minimal, we can find be $(j,J)\in A$ such that $i\in J$ and $(i,I)\in B$ such that $j\in B$.
Since $A\le B$, using Proposition~\ref{prop:sourceorder}, we must have that $S(I,A)\le S(I,B)=i$. In this case it must be that $S(I,A)<\min(J)$, otherwise it would be a cycle.
Similarly, $S(J,B)\ge S(J,A)=j$ implies $S(J,B)>\max(I)$. We habe that $\II$ is closed under intersections, therefore $K=I\cap J\in \II$. Remark that $[i,j]\subseteq K$.
In $A$, we must have $S(K,A)=j$ otherwise we would get a cycle with $(j,J)$. Similarly in $B$ we must have $S(K,B)=i$. The situation is depicted as follow.
$$
\begin{array}{lcr}
	A =  
	\begin{tikzpicture}[scale=1,baseline=.0cm]
	\node at (-.5,.3) {$\scriptstyle J$}; \draw [thick,{Bar[width=3pt]}-] (.5,.3)--(2.2,.3);   \node at (1.7,.3) {$\bullet$};
									  \node at (2.3,.3) {$\scriptstyle \cdot$};\node at (2.4,.3) {$\scriptstyle \cdot$};\node at (2.5,.3) {$\scriptstyle \cdot$};
									  \node at (-.1,0) {$\scriptstyle \cdot$};\node at (0,0) {$\scriptstyle \cdot$};\node at (-.2,0) {$\scriptstyle \cdot$};
	\node at (-.5,0) {$\scriptstyle I$};  \draw [thick,-{Bar[width=3pt]}] (.1,0)--(2,0);   \node at (.3,0) {$\bullet$};
	\node at (-.5,-.3) {$\scriptstyle I\cap J$};  \draw [thick,{Bar[width=3pt]}-{Bar[width=3pt]}] (.5,-.3)--(2,-.3);   \node at (1.7,-.3) {$\bullet$};
	\node at  (.7,-.6) {$\scriptstyle i$};
	\node at  (1.7,-.6) {$\scriptstyle j$};
	\end{tikzpicture} 
	&\qquad\quad&
	\begin{tikzpicture}[scale=1,baseline=.0cm]
	\node at (-.5,.3) {$\scriptstyle J$}; \draw [thick,{Bar[width=3pt]}-] (.5,.3)--(2.2,.3);   \node at (2.2,.3) {$\bullet$};
									  \node at (2.3,.3) {$\scriptstyle \cdot$};\node at (2.4,.3) {$\scriptstyle \cdot$};\node at (2.5,.3) {$\scriptstyle \cdot$};
									  \node at (-.1,0) {$\scriptstyle \cdot$};\node at (0,0) {$\scriptstyle \cdot$};\node at (-.2,0) {$\scriptstyle \cdot$};
	\node at (-.5,0) {$\scriptstyle I$};  \draw [thick,-{Bar[width=3pt]}] (.1,0)--(2,0);   \node at (.7,0) {$\bullet$};
	\node at (-.5,-.3) {$\scriptstyle I\cap J$};  \draw [thick,{Bar[width=3pt]}-{Bar[width=3pt]}] (.5,-.3)--(2,-.3);   \node at (.7,-.3) {$\bullet$};
	\node at  (.7,-.6) {$\scriptstyle i$};
	\node at  (1.7,-.6) {$\scriptstyle j$};
	\end{tikzpicture} 
	=B 
\end{array}
$$
Now we have $S(K,A)=j>i=S(K,B)$ and Proposition~\ref{prop:sourceorder} gives that $A\not\le B$. A contradiction concluding our proof.
\end{proof}


\begin{remark} In Theorem~\ref{thm:propertieofintI} the hypothesis that $\II$ is closed under intersection is used only to prove (a)$\implies$(c).
In general, this implication is not true as shown in the following example. 
$$ \begin{array}{ccccc}
	\begin{tikzpicture}[scale=1,baseline=.2cm]
	\foreach \x in {1,...,4}
		\node (\x) at (\x*.5,-.1) [inner sep = -1pt] {$\scriptstyle \x$};
	\draw [thick,{Bar[width=3pt]}-{Bar[width=3pt]}] (.5,.4)--(1.5,.4);   \node at (.5,.4) {$\bullet$};
	\draw [thick,{Bar[width=3pt]}-{Bar[width=3pt]}] (1,.2)--(2,.2);   \node at (1.5,.2) {$\bullet$};
	\end{tikzpicture}
     &\qquad \le \qquad\qquad&
	\begin{tikzpicture}[scale=1,baseline=.2cm]
	\foreach \x in {1,...,4}
		\node (\x) at (\x*.5,-.1) [inner sep = -1pt] {$\scriptstyle \x$};
	\draw [thick,{Bar[width=3pt]}-{Bar[width=3pt]}] (.5,.4)--(1.5,.4);   \node at (.5,.4) {$\bullet$};
	\draw [thick,{Bar[width=3pt]}-{Bar[width=3pt]}] (1,.2)--(2,.2);   \node at (2,.2) {$\bullet$};
	\end{tikzpicture}     
     &\qquad \le \qquad\qquad&
	\begin{tikzpicture}[scale=1,baseline=.2cm]
	\foreach \x in {1,...,4}
		\node (\x) at (\x*.5,-.1) [inner sep = -1pt] {$\scriptstyle \x$};
	\draw [thick,{Bar[width=3pt]}-{Bar[width=3pt]}] (.5,.4)--(1.5,.4);   \node at (1,.4) {$\bullet$};
	\draw [thick,{Bar[width=3pt]}-{Bar[width=3pt]}] (1,.2)--(2,.2);   \node at (2,.2) {$\bullet$};
	\end{tikzpicture}     \\ \\
     A&&B&&C
    \end{array}$$
Here $[\pi_A^0,\pi_A^1]=[1324,1342]$, $[\pi_B^0,\pi_B^1]=[1423,4132]$ and $[\pi_C^0,\pi_C^1]=[4213,4231]$, but $\pi_A^0\not\le \pi_C^1$.
\end{remark} 

%%%%%%%%%%%%%%%%%%%%%%%%%%%%%%%%%%%%%%
\subsection{If $\II$ is closed under intersections, then $P_\II$}  
\label{subsec:latticeII}

We now conclude the proof of  Theorem~\ref{thm:latticeI}. This is a corollary of Theorem~\ref{thm:propertieofintI}.

\begin{corollary}\label{cor:lattice}
For $\II$ closed under intersection, we have $P_\II$ is a lattice where
$$ A\meet B =\Or_{\pi_A^0\meet \pi_B^0} \qquad\text{ and }\qquad A\join B =\Or_{\pi_A^0\join \pi_B^0}\,.$$
\end{corollary}

\begin{proof} Assume we have $A,B,C,D\in P_\II$ such that $A\le C$, $A\le D$, $B\le C$ and $B\le D$.
Theorem~\ref{thm:propertieofintI} implies that $\pi_A^0\le \pi_C^1$, $\pi_A^0\le \pi_D^1$, $\pi_B^0\le \pi_C^1$ and $\pi_B^0\le \pi_D^1$.
In the weak lattice, we have $\pi_A^0 \join \pi_B^0$ is such that 
  $$\pi_A^0\le (\pi_A^0 \join \pi_B^0),\quad  \pi_B^0\le (\pi_A^0 \join \pi_B^0),\quad (\pi_A^0 \join \pi_B^0) \le \pi_C^1\quad\text{and}\quad(\pi_A^0 \join \pi_B^0)\le \pi_D^1\,.$$
Using the fact that $\Or$ is order preserving, we get
  $$A\le \Or_{\pi_A^0 \join \pi_B^0},\quad B\le \Or_{\pi_A^0 \join \pi_B^0},\quad \Or_{\pi_A^0 \join \pi_B^0}\le C \quad \text{and}\quad  \Or_{\pi_A^0 \join \pi_B^0}\le D\,.$$
This implies that $A$ and $B$ admit a join $(A\join B) \le \Or_{\pi_A^0 \join \pi_B^0}$. Using Theorem~\ref{thm:propertieofintI} with $A\le (A\join B)$ and $B\le (A\join B)$, we get 
  $ \pi^0_A\join \pi^0_B\le  \pi^1_{A\join B}\,,$
and this implies $(A\join B) \ge \Or_{\pi_A^0 \join \pi_B^0}$. This shows the result for $\join$. The proof of $\meet$ is similar.
\end{proof}

\Nantel{In the note, there is an open question (should we make it a remark here or try to work it out?): A \defn{quasi-congruence} is an order preserving map  $\Psi\colon \fS_n\to P$ 
such that for all $A\in P$ we have $\Psi^{-1}(A)=[\pi_A^0, \pi_A^1]$ is an interval and $A\le B$ implies $\pi_A^0\le \pi_B^1$. In this situation, the proof of Corollary~\ref{cor:lattice} applies
and we conclude that $P$ is a lattice where $A\meet B =\Psi(\pi_A^0\meet \pi_B^0)$ and $A\join B =\Psi(\pi_A^0\join \pi_B^0)$. Can we characterize all semi-congruences of weak order?
}


%%%%%%%%%%%%%%%%%%%%%%%%%%%%%%%%%%%%%%
%%%%%%%%%%%%%%%%%%%%%%%%%%%%%%%%%%%%%%
%%%%%%%%%%%%%%%%%%%%%%%%%%%%%%%%%%%%%%

\section{Characterization of  $P_\II$ that are distributive lattices}
\label{sec:distributive}

In this section, we characterize the lattice $P_\II$ that are distributive. To start, we must have a segment hypergraph $\II$ that is closed under intersections.
Our first task will be to identify the join irreducible of $P_\II$.

%%%%%%%%%%%%%%%%%%%%%%%%%%%%%%%%%%%%%%%
%\subsection{Covers in lattice $P_\II$}  
%\label{subsec:LatticeCover}
%
%In Section~\ref{subsec:cover} we describe cover for general segment hypergraph. If $\II$ is closed under intersection the situation is simpler
%\Nantel{do we need this?}

%%%%%%%%%%%%%%%%%%%%%%%%%%%%%%%%%%%%%%
\subsection{Join irreducibles of lattice $P_\II$}  
\label{subsec:irreducibles}

Fix $\II$ a segment hypergraph closed under intersections. 
Let $O_0$ denote the unique \defn{minimum} of $P_\II$.
Let
  $$\cJ_\II=\bigcup_{I\in \II} I\ssm \{\min(I)\} \,.$$
For $j\in \cJ_\II$ let
\begin{equation}\label{eq:mu}
 \mu(j)=\max\big\{ \min(I) \big| \ I\in\II, j\in I\ssm\{\min(I)\}\big\}\,,
\end{equation}
\begin{equation}\label{eq:nu}
 \nu(j)=\min\big\{ \max(I) \big| \ j\in I\in\II, \min(I)=\mu(j)\big\}\,,
\end{equation}
and
\begin{equation}\label{eq:nu}
 O_j=\Or_{(\mu(j),\mu(j)+1,\ldots,j)}\,,
\end{equation}
where $(\mu(j),\mu(j)+1,\ldots,j)$ is the cycle permutation
  $$(\mu(j),\mu(j)+1,\ldots,j)= 12\cdots(\mu(j)-1)j\,\mu(j)\cdots (j-1) (j+1)\cdots n
  $$
with $j$ place just before $\mu(j)$ and all others in order. We remark that for all $I\in\II$ and $j\in\cJ_\II$
\begin{equation}\label{eq:sourceOj}
 	S(I,O_j)=\begin{cases}
		j& \text{if $j\in I$ and $\min(I)=\mu(j)$},\\
		\min(I)& \text{otherwise.}
	\end{cases}
\end{equation}

\begin{remark}\label{rem:munuseg}
For any $j\in \cJ_\II$ and $j\in J\in \II$ such that $j\ne\min(J)$, we have $[\mu(j),\nu(j)]\in \II$ and $[\mu(j),\nu(j)]\subseteq J$.
Indeed, the definition of $\mu(j)$ and $\nu(j)$ ensure that $[\mu(j),\nu(j)]\in \II$, and if we suppose that $[\mu(j),\nu(j)]\not\subseteq J$, then $[\mu(j),\nu(j)]\cap J\in \II$ is strictly smaller than $[\mu(j),\nu(j)]$ a contradiction to the definition of $\mu$ and $\nu$.
\end{remark}

For $i,j\in \cJ_\II$ we say $i\equiv j$ are \defn{irreducible congruent} if $\mu(i)=\mu(j)$ and $\nu(i)=\nu(j)$.

\begin{lemma} \label{lem:irrorder}
 For $i,j\in \cJ_\II$, we have $O_i<O_j$ if and only if $i<j$ and $i\equiv j$.
\end{lemma}

\begin{proof} Proposition~\ref{prop:sourceorder} gives us that $O_i<O_j$ if and only if
 	$$ S(I,O_i)=\begin{cases}
		i<j & \text{if $j\in I$ and $\min(I)=\mu(j)$},\\
		\min(I)& \text{otherwise.}
	\end{cases}$$
which is true if $i\equiv j$. Now if $i\not\equiv j$, then either $\mu(i)\ne \mu(j)$  or  $\nu(i)\ne \nu(j)$.
If $\mu(i)\ne\mu(j)$, then there is $i\in I\in \II$ such that $\min(I)=\mu(i)\ne\mu(j)$ and
  $S(I,O_i)=i>\min(I)=S(I,O_j)$. This implies $O_i\not<O_j$ a contradiction. If $\mu(i)= \mu(j)$  and  $\nu(i)\ne \nu(j)$,
  then there is $i\in I\in \II$ such that $j\not\in I$, and again $S(I,O_i)=i>\min(I)=S(I,O_j)$, a contradiction.
\end{proof} 

\begin{lemma}\label{lem:irrOj}
	For any $j\in \cJ_\II$, $O_j$ is join irreducible.
\end{lemma}

\begin{proof} Proposition~\ref{prop:sourceorder} gives that any $A<O_j$ must satisfy
 	$$ S(I,A)=\begin{cases}
		a<j & \text{if $j\in I$ and $\min(I)=\mu(j)$},\\
		\min(I)& \text{otherwise.}
	\end{cases}$$
Remark~\ref{rem:munuseg} ensure that $a$ must be the same for all $I\ni j$ and $\min(I)=\mu(j)$, otherwise $A$ would not be acyclic.
$A=O_0$ is one possibility, otherwise $A=O_a$ for some $0<a<j$. In Lemma~\ref{lem:irrorder} we determined exactly the order $P_\II$ restricted
to the set $\{ O_j |\, j=0 \text{ or } j\in \cJ_\II\}$. We have shown  that $A<O_j$ belong to that set hence the only cover is $O_a<O_j$
such that $a=\max\{ i<j|\, i=0 \text{ or } i\equiv j\}$.
\end{proof}

\begin{lemma}\label{lem:subirr}
	For any $A\in P_\II$,
	 $$ O_j<A \iff j\le S([\mu(j),\nu(j)] A). $$
\end{lemma}

\begin{proof}
This is a direct consequence of Proposition~\ref{prop:sourceorder} and Remark~\ref{rem:munuseg}.
\end{proof}

\begin{proposition} Assume that
 ``given any $I,J\in \II$ such that $I\cap J\ne \emptyset$, $I\not\subseteq J$ and $J\not\subseteq I$,
 then for all $K\in J$ such that  $I\cap J\subseteq K$ we have $\{\min(K),\max(K)\}\cap K\ne \emptyset$'',
 then $\{O_j|\,j\in \cJ_\II\}$ are the only join irreducible of $P_\II$. 
\end{proposition}

\begin{proof}
   Let $O$ be a joint irreducible of $P_\II$ and let $A<O$ be its unique cover. It must be from a flip $O=(i,j)A$.
   \nantel{that is all folks}
\end{proof}

%%%%%%%%%%%%%%%%%%%%%%%%%%%%%%%%%%%%%%
%%%%%%%%%%%%%%%%%%%%%%%%%%%%%%%%%%%%%%
%%%%%%%%%%%%%%%%%%%%%%%%%%%%%%%%%%%%%%

\section*{Acknowledgments}

Thanks!
\Nantel{Still to do: Finish Section 5,   description of $\join$ and $\meet$ in term of sources(?), characterize which Segment hypergraph gives lattice quotient(?), semi-distributive, what else?}

%%%%%%%%%%%%%%%%%%%%%%%%%%%%%%%%%%%%%%
%%%%%%%%%%%%%%%%%%%%%%%%%%%%%%%%%%%%%%
%%%%%%%%%%%%%%%%%%%%%%%%%%%%%%%%%%%%%%

\bibliographystyle{alpha}
\bibliography{IH_lattices}
\label{sec:biblio}

%%%%%%%%%%%%%%%%%%%%%%%%%%%%%%%%%%%%%%

\end{document}
