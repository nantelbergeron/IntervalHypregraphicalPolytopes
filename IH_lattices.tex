\documentclass[reqno]{amsart}

\usepackage{enumerate, amsmath, amsfonts, amssymb, amsthm, wasysym, graphics, graphicx, xcolor, url, hyperref, hypcap, a4wide, pdflscape, multido, xargs, colortbl, multicol, multirow, calc, shuffle, hvfloat}
\hypersetup{colorlinks=true, citecolor=PineGreen, linkcolor=PineGreen}
%\usepackage[all]{xy}
\usepackage{tikz}\usetikzlibrary{trees,snakes,shapes,arrows,matrix,calc,arrows.meta}
\usepackage{comment}
\usepackage{etex}
\usepackage{ulem}\normalem % to strike through a word
\usepackage[noabbrev,capitalise]{cleveref}
\setlength{\abovecaptionskip}{10pt}
\setlength{\belowcaptionskip}{5pt}

%\reserveinserts{50}
\graphicspath{{figures/}}
\makeatletter
\def\input@path{{figures/}}
\makeatother

%%%%%%%%%%%%%%%%%%%%%%%%%%%%%%%%%%%%%%

\title{Lattices of Segment Hypergraphic Polytopes}

\author[N.~Bergeron]{Nantel Bergeron} 
\address[N.~Bergeron]{Department of Mathematics and Statistics, York University, Toronto}
\email{bergeron@yorku.ca}
\urladdr{http://bergeron.mathstats.yorku.ca}

\author[V.~Pilaud]{Vincent Pilaud}
\address[V.~Pilaud]{??? CNRS \& LIX, \'Ecole Polytechnique, Palaiseau}
\email{vincent.pilaud@lix.polytechnique.fr}
\urladdr{http://www.lix.polytechnique.fr/~pilaud/}

\thanks{
NB was supported by NSERC and York Research Chair in Applied Algebra.
VP ???).
}



%%%%%%%%%%%%%%%%%%%%%%%%%%%%%%%%%%%%%%

% theorems
\newtheorem{theorem}{Theorem}[section]
\newtheorem{theoremA}{Theorem}
\renewcommand{\thetheoremA}{\Alph{theoremA}}
\newtheorem{corollary}[theorem]{Corollary}
\newtheorem{proposition}[theorem]{Proposition}
\newtheorem{lemma}[theorem]{Lemma}
\newtheorem{conjecture}[theorem]{Conjecture}
\crefname{conjecture}{Conjecture}{Conjectures}
\newtheorem{conjectureA}{Conjecture}
\renewcommand{\theconjectureA}{\Alph{conjectureA}}
\crefname{conjectureA}{Conjecture}{Conjectures}

\theoremstyle{definition}
\newtheorem{definition}[theorem]{Definition}
\newtheorem{example}[theorem]{Example}
\newtheorem{remark}[theorem]{Remark}
\newtheorem{question}[theorem]{Question}
\newtheorem{notation}[theorem]{Notation}
\newtheorem{openproblem}[theorem]{Open problem}


% newcommands
% math special letters
\newcommand{\R}{\mathbb{R}} % reals
\newcommand{\N}{\mathbb{N}} % naturals
\newcommand{\Z}{\mathbb{Z}} % integers
\newcommand{\I}{\mathbb{I}} % set of integers
\newcommand{\C}{\mathbb{C}} % set of summands
\renewcommand{\b}[1]{\boldsymbol{#1}} % bold
\newcommand{\cal}[1]{\mathcal{#1}} % cal

% math commands
\newcommand{\set}[2]{\left\{ #1 \;\middle|\; #2 \right\}} % set notation
\newcommand{\bigset}[2]{\big\{ #1 \;|\; #2 \big\}} % big set notation
\newcommand{\biggset}[2]{\bigg\{ #1 \;\bigg|\; #2 \bigg\}} % big set notation
\newcommand{\multiset}[2]{\left\{\!\!\left\{ #1 \;\middle|\; #2 \right\}\!\!\right\}} % multiset notation
\newcommand{\bigmultiset}[2]{\big\{\!\!\big\{ #1 \;|\; #2 \big\}\!\!\big\}} % big multiset notation
\newcommand{\ssm}{\smallsetminus} % small set minus
\newcommand{\dotprod}[2]{\langle #1 | #2 \rangle} % dot product
\newcommand{\symdif}{\triangle} % symmetric difference
\newcommand{\one}{{1\!\!1}} % the all one vector
\newcommand{\eqdef}{\mbox{\,\raisebox{0.2ex}{\scriptsize\ensuremath{\mathrm:}}\ensuremath{=}\,}} % :=
\newcommand{\defeq}{\mbox{~\ensuremath{=}\raisebox{0.2ex}{\scriptsize\ensuremath{\mathrm:}} }} % =:
\newcommand{\polar}{^\diamond} % polar
\newcommand{\simplex}{\triangle} % simplex

% operators
\DeclareMathOperator{\conv}{conv} % convex hull
\DeclareMathOperator{\cone}{cone} % cone hull
\DeclareMathOperator{\arr}{Arr} % arrangements
\DeclareMathOperator{\Inv}{Inv} % inversion set
\DeclareMathOperator{\Ninv}{Ninv} % non-inversion set
\DeclareMathOperator{\DemazureProduct}{Dem} % non-inversion set

% others
\newcommand{\fix}[1]{{\bf FIXME: }#1} % emphasis of a problem to FIX
\newcommand{\ie}{\textit{i.e.}~} % id est
\newcommand{\eg}{\textit{e.g.}~} % exempli gratia
\newcommand{\Eg}{\textit{E.g.}~} % exempli gratia
\newcommand{\aka}{\textit{aka.}~} % also known as
\newcommand{\viceversa}{\textit{vice versa}} % vice versa
\newcommand{\ordinal}{\textsuperscript{th}} % th for ordinals
\newcommand{\ex}[1]{^{\textrm{ex#1}}} % example
\newcommand{\para}[1]{\medskip\noindent\textbf{#1}} % paragraph
\newcommand{\subpara}[1]{\smallskip\noindent\textit{#1.}} % paragraph
\definecolor{PineGreen}{RGB}{2,120,120} % pinegreen color
\definecolor{darkgreen}{RGB}{57,181,74} % darkgreen color
\newcommand{\blue}[1]{{\color{blue} #1}} % blue
\newcommand{\red}[1]{{\color{red} #1}} % red
\newcommand{\green}[1]{{\color{darkgreen} #1}} % green
\newcommand{\defn}[1]{\textbf{\textsf{\color{PineGreen} #1}}} % emphasis of a definition
\usepackage{todonotes}
\newcommand{\nantel}[1]{\todo[color=red!30]{#1 \\ \hfill --- N.}}
\newcommand{\Nantel}[1]{\todo[inline,color=red!30]{#1 \\ \hfill --- N.}}
\newcommand{\vincent}[1]{\todo[color=blue!30]{#1 \\ \hfill --- V.}}
\newcommand{\Vincent}[1]{\todo[inline,color=blue!30]{#1 \\ \hfill --- V.}}

% permutations
\newcommand{\fS}{\mathfrak{S}} % symmetric group
\newcommand{\fR}{\mathfrak{R}} % subset symmetric group


% lattices
\newcommand{\meet}{\wedge} % meet
\newcommand{\join}{\vee} % join
\newcommand{\less}{\vartriangleleft} % smaller WOIP
\newcommand{\more}{\vartriangleright} % larger WOIP
\newcommand{\contactLess}[1]{\less_{#1}} % smaller contact graph
\newcommand{\contactMore}[1]{\more_{#1}} % larger contact graph
\newcommand{\projDown}{\pi_\downarrow} % Down projection
\newcommand{\projUp}{\pi^\uparrow} % Down projection

% Orientation, Hypergraph and Segments
\newcommand{\Or}{\mathcal O}  % acyclic orientation for hypergraph (vertices for us)
\newcommand{\HH}{\mathbb H}  % general hypergraph
\newcommand{\II}{\mathbb I} % segment hypergraph


%%%%%%%%%%%%%%%%%%%%%%%%%%%%%%%%%%%%%%
%%%%%%%%%%%%%%%%%%%%%%%%%%%%%%%%%%%%%%
%%%%%%%%%%%%%%%%%%%%%%%%%%%%%%%%%%%%%%

\begin{document}

\begin{abstract}
	Lets be real, not just abstract
\end{abstract}

\vspace*{-.8cm}

\maketitle
\nantel{Todo ``Nantel" avec majuscule sont en ligne avec le texte, "nantel" est comme avant}
\Nantel{Vincent, verifie ton affiliation, grant et addresse}


\tableofcontents

%%%%%%%%%%%%%%%%%%%%%%%%%%%%%%%%%%%%%%
%%%%%%%%%%%%%%%%%%%%%%%%%%%%%%%%%%%%%%
%%%%%%%%%%%%%%%%%%%%%%%%%%%%%%%%%%%%%%

\section{Introduction}
\label{sec:introduction}

\begin{theoremA}\label{thm:latticeI}
For a segment hypergraph $\II$ the poset $P_\II$ is a lattice if and only if $\II$ is closed under intersections.
\end{theoremA}

%%%%%%%%%%%%%%%%%%%%%%%%%%%%%%%%%%%%%%
%%%%%%%%%%%%%%%%%%%%%%%%%%%%%%%%%%%%%%
%%%%%%%%%%%%%%%%%%%%%%%%%%%%%%%%%%%%%%

\section{Hypergraphical Polytopes and their partial orders}
\label{sec:HP}


%%%%%%%%%%%%%%%%%%%%%%%%%%%%%%%%%%%%%%
\subsection{Hypergraphic Polytope} 
\label{subsec:D_H}

Let $[n]=\{1,2,\ldots,n\}$. A \defn{hypergraph} $\HH$ on $[n]$, is a collection of  subsets of $[n]$.
By convention, we will assume that for all $i\in [n]$, we have $\{i\}\in \HH$.
Following~\cite[Def 2.11]{BenBerMac}, given an hypergraph $\HH$, we define the \defn{hypergraphic polytope}
$\Delta_{\HH}$ as the the Minkowski sum
 $$\Delta_{\HH} = \sum_{H\in \HH} \Delta_H\,,$$
where $\Delta_H$ is the simplex given by the convex hull of the points $\{e_i | i\in H\}\subset \R^n$.

\begin{example}\label{ex:DH}
For the hypergraph
$\HH=\big\{\{1\},\{2\},\{3\},\{4\},\{3,4\},\{1,2,3\}\big\}$,
we  have
$$\begin{array}{ccc}
 \begin{tikzpicture}[scale=1,baseline=.5cm]
	\node (1) at (0.6,1.0) {$\scriptstyle e_1$};
	\node (2) at (-.2,-.2) {$\scriptstyle e_2$};
	\node (3) at (1.5,-.2) {$\scriptstyle e_3$};
	\draw [fill=blue!40] (0,0) -- (.6,.75) -- (1.2,0) --(0,0) ; 
\end{tikzpicture} \quad &
 \begin{tikzpicture}[scale=1,baseline=.5cm]
	\node at (-.2,0) {$\scriptstyle e_3$};
	\node at (1.2,.5) {$\scriptstyle e_4$};
	\draw [thick,color=red] (0,0) -- (1,.5 ); 
\end{tikzpicture} \quad &
\begin{tikzpicture}[scale=1,baseline=.5cm]
	\draw [fill=gray!10] (0,0) -- (.6,.75)-- (1.6,1.25) -- (2.2,.5) -- (1.2,0) --(0,0) ; 
	\draw [color=gray!10,fill=blue!20] (0,0) -- (.6,.75) -- (1.2,0) --(0,0) ; 
	\draw [dotted,color=red] (0,0)--(1,.5);
	\draw [dotted,color=blue] (1.6,1.25)--(1,.5)--(2.2,.5);
	\draw (0,0) -- (.6,.75)-- (1.6,1.25) -- (2.2,.5) -- (1.2,0) -- (0,0) ; 
	\draw [thick,color=red!80] (1.2,0) -- (2.2,0.5); 
\end{tikzpicture}\\
\blue{{ \Delta}_{123}}& \red{{ \Delta}_{34}} & \Delta_{\HH}=\Delta_1+\Delta_2+\Delta_3+\Delta_4+\blue{{ \Delta}_{123}}+ \red{{ \Delta}_{34}}\\
\end{array}
$$
which is a 3-dimensional polytope sitting in $\R^4$. 
\end{example}

\begin{remark}\label{rem:single} In~\cite{BenBerMac}, the authors always assume that for all $i\in[n]$ we have $\{i\}\not\in\HH$ whereas here we took the opposite convention. 
When quoting result from~\cite{BenBerMac}, the reader will have to be mindful of this difference.
We point out, as in  Example~\ref{ex:DH}, that the $\Delta_i$ are single points and in the definition of $\Delta_{\HH}$ it simply translate the polytope in the direction of $e_i$ but does not affect the  face structure of the polytope.
\end{remark}
%%%%%%%%%%%%%%%%%%%%%%%%%%%%%%%%%%%%%%
\subsection{Hypergraphic Posets} 
\label{subsec:P_H}
One of the main result of~\cite[Thm 3.18]{BenBerMac} is to identify the faces of $\Delta_\HH$ with the acyclic orientations of certain quotient of $\HH$.  A useful tool,~\cite[Lem 2.9]{BenBerMac}, gives us a surjective map $\Omega$ from the set composition (ordered set partitions) of $[n]$ to the set of faces of $\Delta_\HH$. 
 In the present work, we will be interested in the skeleton (faces of dimension 0 and 1)
of the polytope as an oriented graph $P_{\HH}$. It will turn out that $P_{\HH}$ is a poset (partial order).  The map $\Omega$ identifies the vertices (0-dim faces) $V_{\HH}$  of $\Delta_{\HH}$ as the image of the set compositions of $[n]$ with $n$ parts. In turn, the set composition with $n$ parts correspond to ordered list of $[n]$ and can be encoded with $\fS_n$ the permutations of the set $[n]$. This gives us a map $\Omega_0 \colon \fS_n \to V_{\HH}$.
More explicitly, given $\pi\in\fS_n$ we construct an acyclic orientation $\Or_\pi$ of $\HH$ as follow. For any $H\in \HH$ such that $|H|>1$ we orient $H$  with the source $i$  the leftmost element of $H$ in $\pi$. That is
\begin{equation}\label{eq:orientation}
	\Or_\pi = \Big\{\big(i,H\big)\ \big| \ H\in \HH, |H|>1,  i=\pi\big(\min\{\pi(j): \pi(j)\in H\}\big)\Big\}\,.
\end{equation}
We immediately obtain the following lemma.

\begin{lemma}\label{lem:Hvertices}
For any hypergraph $\HH$, the set of vertices of $\Delta_{\HH}$ is given by $V_{\HH}=\{\Or_\pi: \pi\in \fS_n\}$.
\end{lemma}

We now turn our attention to the edges (1-dim faces) $E_{\HH}$ of $\Delta_{\HH}$. The map $\Omega$ identifies the set $E_{\HH}$ as certain non-trivial image of the set compositions of $[n]$ with $n-1$ parts.
More precisely, let $A=(A_1,A_2,\ldots,A_{n-1})$ be a set composition of $[n]$ with $n-1$ parts. There is a unique part $A_i=\{a_i,b_i\}$ of size two and all other parts $A_j=\{a_j\}$ for $j\neq i$ of size one. It is known~\cite{???} that $A$ 
corresponds  exactly to the cover relation $\sigma=a_1\cdots a_{i-1}a_ib_ia_{i+1}\cdots a_{n-1} < a_1\cdots a_{i-1}b_ia_ia_{i+1}\cdots a_{n-1}=\pi$ of the (right) weak order on $\fS_n$, where $a_i<b_i$. 
Hence given a cover $\sigma < \pi$ in the $\fS_n$-weak  order, two cases may happen. On one hand it is possible that $\Or_\sigma=\Or_\pi$, hence $\Omega(A)=\Or_\pi$ is a vertex and not an edge. On the other hand, if we get the (non-trivial)
relation $\Or_\sigma\ne \Or_\pi$, then $\Omega(A)$ is an (oriented) edge $\Or_\sigma\to\Or_\pi$ of $\Delta_{\HH}$. We thus have the following lemma.

\begin{lemma}\label{lem:Hedges}
For any hypergraph $\HH$, the set of (oriented) edges of $\Delta_{\HH}$ is given by 
 $$E_{\HH}=\{(\Or_\sigma,\Or_\pi): \sigma<\pi \text{ is a cover}, \Or_\sigma\ne\Or_\pi\}\,.$$
\end{lemma}

\begin{proposition}\label{prop:PHisOrder}
The oriented graph $P_{\HH}=(V_{\HH},E_{\HH})$ is acyclic, in particular it defines a poset on $V_{\HH}$.
\end{proposition}

\begin{proof} It suffice to remark that any cycle on $P_{\HH}$ would induce a cycle on the $\fS_n$-weak order, a contradiction.
\end{proof}

The previous proposition allows us to define the \defn{hypergraphic poset} $P_{\HH}$ of any hypergraph $\HH$.  
We can summarize the previous result with the following proposition

\begin{proposition}\label{prop:WeakToP}
With the weak order on $\fS_n$, the map $\Omega_0 \colon \fS_n \to P_{\HH}$ is order preserving.
\end{proposition}



\begin{remark}\label{rem:EdgeNotCover}
It is important to notice that the edges in $E_{\HH}$ are not necessarily covers of the relation in $P_{\HH}$. The simplest example is given by $\HH=\big\{\{1\},\{2\},\{3\},\{1,2,3\}\big\}$.
The vertices $V_{\HH}$ of the polytope $\Delta_{\HH}$ are given by
	$$\Big\{ \Or_{123}=\Or_{132}, \Or_{213}=\Or_{231}, \Or_{312}=\Or_{321}\Big\}= \Big\{ \big\{(1,\{1,2,3\})\big\},\big\{(2,\{1,2,3\})\big\},\big\{(3,\{1,2,3\})\big\}\Big\}$$
and the edges
	$$ E_{\HH}=\big\{ (\Or_{123},\Or_{213}),(\Or_{132},\Or_{312}),(\Or_{231}, \Or_{312})\big\}\,.$$
in picture, that gives
$$\Delta_\HH=\Delta_1+\Delta_2+\Delta_3+\Delta_{123}=
\begin{tikzpicture}[scale=.7,baseline=.0cm]
	\node (a) at (0,-1.2) {$\scriptscriptstyle \{(1,\{1,2,3\})\}$};
	\node (b) at (3,0) {$\scriptscriptstyle \{(2,\{1,2,3\})\}$};
	\node (c) at (0,1.2) {$\scriptscriptstyle \{(3,\{1,2,3\})\}$};
	\draw [color=blue!40,thick,fill=blue!20] (0,-1)--(0,1)--(1.632,0)--(0,-1) ; 
	\draw [color=red,thick,->] (-.1,-1)--(-.1,1); 
	\draw [color=red,thick,->] (1.832,0)--(.2,1); 
	\draw [color=red,thick,->] (.2,-1)--(1.832,0); 
\end{tikzpicture}
$$
and clearly the edge $(\Or_{132},\Or_{312})$ is not a cover. 
\end{remark}

%%%%%%%%%%%%%%%%%%%%%%%%%%%%%%%%%%%%%%
\subsection{Sources and preimage of $\Omega_0$} 
\label{subsec:notation}

We have seen that the map $\Omega_0 \colon \fS_n \to V_{\HH}$ is surjective where $\Omega_0(\pi)=\Or_\pi$.
We now set some notation in this context. Let $\HH$ be  an hypergraph on $[n]$. Given a permutation $\pi\in\fS_n$ and $H\in \HH$,
we say that 
\begin{equation}\label{eq:source}
	S(H,\Or_\pi)=\pi\big( \min\{j:\pi(j)\in H\}\big)
\end{equation}
is the \defn{source} of $H$ for the orientation $\Or_\pi$. 

Given an acyclic orientation $\Or$ of $\HH$,  we can characterize the preimage 
	$$ \Omega_0^{-1}(\Or)=\{ \pi : \Or_\pi=\Or\}$$
as follow. Let $\less_\Or$ be the order on $[n]$ defined by the transitive closure of the union of 
the order $\less_{(i,H)}=\big\{ i< j : j\in H\ssm\{i\}\big\}$ for each $(i,H)\in \Or$ . That is
 	$$\less_\Or =  \bigcup_{(a,H)\in \Or} 
	\begin{tikzpicture}[scale=1,baseline=.0cm]
	\node at (0,-.45) {$\scriptstyle i$};
	\node at (0,.6) {$\scriptstyle j\in H\ssm \{i\}$};
	\node at (.2,.35) {$\ldots$};
	\draw [thick] (0,-.3)--(-.5,.4); 
	\draw [thick] (0,-.3)--(-.3,.4); 
	\draw [thick] (0,-.3)--(-.1,.4); 
	\draw [thick] (0,-.3)--(.5,.4); 
	\end{tikzpicture}
	$$
This is a well defined order since $\Or$ is acyclic. The following is a straightforward lemma left to the reader to prove.
\begin{lemma}\label{lem:prepi}   $ \Omega_0^{-1}(\Or) =\big\{ \pi : \pi \text{ is a linear extension of }  \less_\Or)\big\}$.
\end{lemma}


In General, given a hypergraph $\HH$ we would be interested to understand what are the cover of $P_{\HH}$? 
When is $P_{\HH}$ a lattice? distributive lattice? a semi-lattice? (semi-)lattice quotient of the $fS_n$-weak lattice?
Can we have a better description of $\Omega_0^{-1}$? and more.

In the present work we focus our attention to a subfamily of Hypergraph, namely we assume that all $H\in \HH$ are segment.

%%%%%%%%%%%%%%%%%%%%%%%%%%%%%%%%%%%%%%
%%%%%%%%%%%%%%%%%%%%%%%%%%%%%%%%%%%%%%
%%%%%%%%%%%%%%%%%%%%%%%%%%%%%%%%%%%%%%

\section{Segment Hypergraphic Polytopes and their partial orders}
\label{sec:IHP}

In this paper, we  assume that $\HH=\II$ is an hypergraph where all $I\in \II$ are segment of the form $I=[a,b]=\{a,a+1,a+2,\ldots,b\}$.
We call such hypergraph a \defn{segment hypergprah}. 

\begin{example}\label{ex:seghyp}
 $\II=\{\{1\},\{2\},\{3\},\{4\},[1,3],[2,3],[2,4],[1,4]\}$ is a segment hypergraph.
We will represent $\II$ and an orientation $\Or_{4132} =\big\{ (4,\{1,2,3,4\}), (1,\{1,2,3\}),(3,\{2,3\}),(4,\{2,3,4\})\big\}$ of $\II$ graphicaly as follow:
 	$$\II =  
	\begin{tikzpicture}[scale=1,baseline=.0cm]
	\foreach \x in {1,...,4}
		\node (\x) at (\x*.5,-.4) [inner sep = -1pt] {$\scriptstyle \x$};
	\draw [thick,{Bar[width=3pt]}-{Bar[width=3pt]}] (.5,-.2)--(2,-.2); 
	\draw [thick,{Bar[width=3pt]}-{Bar[width=3pt]}] (.5,0)--(1.5,0); 
	\draw [thick,{Bar[width=3pt]}-{Bar[width=3pt]}] (1,.2)--(1.5,.2); 
	\draw [thick,{Bar[width=3pt]}-{Bar[width=3pt]}] (1,.4)--(2,.4); 
	\end{tikzpicture}
	\qquad\qquad
	\Or_{4132}  =  
	\begin{tikzpicture}[scale=1,baseline=.0cm]
	\foreach \x in {1,...,4}
		\node (\x) at (\x*.5,-.4) [inner sep = -1pt] {$\scriptstyle \x$};
	\draw [thick,{Bar[width=3pt]}-{Bar[width=3pt]}] (.5,-.2)--(2,-.2);  \node at (2,-.2) {$\bullet$};
	\draw [thick,{Bar[width=3pt]}-{Bar[width=3pt]}] (.5,0)--(1.5,0);   \node at (.5,0) {$\bullet$};
	\draw [thick,{Bar[width=3pt]}-{Bar[width=3pt]}] (1,.2)--(1.5,.2);   \node at (1.5,.2) {$\bullet$};
	\draw [thick,{Bar[width=3pt]}-{Bar[width=3pt]}] (1,.4)--(2,.4);   \node at (2,.4) {$\bullet$};
	\end{tikzpicture}
	$$
and we omit to draw the singleton $\{i\}$ for $1\le i\le n$.
\end{example}

%%%%%%%%%%%%%%%%%%%%%%%%%%%%%%%%%%%%%%
\subsection{Preimage $\Omega_0^{-1}$ for segment hypergraph} 
\label{subsec:preimageI}
One striking property for segment hypergraph is that  the sets  $\Omega_0^{-1}(\Or)$ have very nice properties.
To describe this we first need to recall a classic result about linear extension of orders on $[n]$.

\begin{proposition}[{\cite[Thm.~6.8]{BjornerWachs}}]
\label{prop:WOIP}
The set of linear extensions of a poset~$\less$ on~$[n]$ forms an interval~$I$ of the weak order if and only if for every~$i < j < k$,
\[
i \less k \implies i \less j \text{ or } j \less k
\qquad\text{and}\qquad
i \more k \implies i \more j \text{ or } j \more k.
\]
Moreover, the inversions of~$\min(I)$ are the pairs~$i,j \in [n]$ with $i < j$ and $i \more j$, and the non-inversions of~$\max(I)$ are the pairs~$i,j \in [n]$ with $i < j$ and $i \less j$.
\end{proposition}


\begin{proposition}\label{prop:preimage}
 Given a segment hypergraph $\II$, we have that for any vertex in $V_{\II}$ describe by an orientation $\Or$, the set $\Omega_0^{-1}(\Or)$ is an interval 
 in the weak order with minimum avoiding the pattern $231$ and maximum avoiding the pattern $213$.
\end{proposition}

\begin{proof}
From Lemma~\ref{lem:prepi} we have that any $\pi\in \Omega_0^{-1}(\Or)$  is a linear extension of $\less_\Or$, we can therefor use Proposition~\ref{prop:WOIP} to prove our claim. 
Let $i<j<k$ be any integers in the natural order of $[n]$. If we have $i\less_{\Or} k$, then, by definition of $\less_\Or$, there must be a sequence $(i_1,I_{1}), (i_2,I_{2}),\ldots (i_m,I_{m})$ of oriented edges in $\Or$ such that
$i_1=i$, for $1\le s<m$ we have $i_{s+1}\in I_s$, and $k\in I_m$. We must therefor have $j\in\bigcup_{1\le s\le m} I_s$ since both $i$ and $k$ are in the union. This implies $i\less_{\Or} j$. The case $i\more_{\Or} k$ is similar and  implies
that $j\more_\Or k$. This shows $\Omega_0^{-1}(\Or)$ is an interval in the weak order.

For the remaining of our claim, assume $\pi$ is the maximum of the interval of $\Omega_0^{-1}(\Or)$ and has a $213$ pattern. That is,  we can find $i<j<k$ such that $\{i,k\}$ is a non-inversion of $\pi$ and $\{i,j\}$ is an inversion. Using Proposition~\ref{prop:WOIP} we must have $i\less_\Or k$ and $i\not\less_\Or j$, a contradiction to our proof above. The case for the minimum is similar.
\end{proof}


%%%%%%%%%%%%%%%%%%%%%%%%%%%%%%%%%%%%%%
\subsection{A source characterization of $P_\II$}  
\label{subsec:sourceinc}

A necessary and sufficient  condition of the order $P_\II$ is that sources must increase for each $I\in \II$.

\begin{proposition}
\label{prop:sourceorder}
Let $\II$ be a segment hypergraph. For any $A,B\in P_\II$ we have
$$ A\le B\quad  \text{ if and only if }\quad  S(I,A)\le S(I,B) \text{ for all } I\in \II .$$
\end{proposition}
\begin{proof}

\end{proof}


%%%%%%%%%%%%%%%%%%%%%%%%%%%%%%%%%%%%%%
%%%%%%%%%%%%%%%%%%%%%%%%%%%%%%%%%%%%%%
%%%%%%%%%%%%%%%%%%%%%%%%%%%%%%%%%%%%%%

\section{Which $P_\II$ are lattices?}
\label{sec:LatticePI}

In this section, we will prove our Theorem~\ref{thm:latticeI} that characterizes the segment hypergraphs $\II$ for which $P_\II$ are lattices.

%%%%%%%%%%%%%%%%%%%%%%%%%%%%%%%%%%%%%%
\subsection{If $P_\II$ is a lattice, then $\II$ is closed under intersections}  
\label{subsec:latticeI}

We are now ready to show the forward implication of Theorem~\ref{thm:latticeI}.

\begin{proposition}
	For a segment hypergraph $\II$, if the poset $P_\II$ is a lattice, then $\II$ is closed under intersections.
\end{proposition}

\begin{proof} By contradiction, assume we have $I,J\in \II$ such that $\emptyset \not = I\cap J\not\in \II$. Let $a<b<c<d$ be such that $a=b-1$, $b=\min(I\cap J)$, $c=\max(I\cap J)$, and $d=c+1$.
Remark that $b\ne c$ since $\II$ contains all singletons. By symmetry, we assume that $a\in I\ssm J$ and $d\in J\ssm I$. Let $X$ be the element of $[n]\ssm\{a,b,c,d\}$ written in increasing order.
We now construct four permutations 
$$ \pi_A=bacdX,\qquad \pi_B=acdbX,\qquad \pi_C=dbacX \qquad\text{and}\qquad \pi_D=cdbaX,$$
and consider the four distinct orientations $A=\Or_{pi_A}$, $B=\Or_{pi_B}$, $C=\Or_{pi_C}$ and $D=\Or_{pi_D}$. We display bellow the four orientations highlighting only the segment $I$ and $J$.
$$
\begin{array}{lcr}
	C =  
	\begin{tikzpicture}[scale=1,baseline=.0cm]
	\node at (-.3,.3) {$\scriptstyle I$}; \draw [thick,{Bar[width=3pt]}-{Bar[width=3pt]}] (0,.3)--(1.2,.3);   \node at (.3,.3) {$\bullet$};
	\node at (-.3,0) {$\scriptstyle J$};  \draw [thick,{Bar[width=3pt]}-{Bar[width=3pt]}] (.3,0)--(1.5,0);   \node at (1.5,0) {$\bullet$};
	\node at  (0,-.3) {$\scriptstyle a$};
	\node at  (.3,-.3) {$\scriptstyle b$};
	\node at  (1.2,-.3) {$\scriptstyle c$};
	\node at  (1.5,-.3) {$\scriptstyle d$};
	\end{tikzpicture} 
	&\qquad\quad&
	\begin{tikzpicture}[scale=1,baseline=.0cm]
	\node at (2,.3) {$\scriptstyle I$}; \draw [thick,{Bar[width=3pt]}-{Bar[width=3pt]}] (0,.3)--(1.2,.3);   \node at (1.2,.3) {$\bullet$};
	\node at (2,0) {$\scriptstyle J$};  \draw [thick,{Bar[width=3pt]}-{Bar[width=3pt]}] (.3,0)--(1.5,0);   \node at (1.2,0) {$\bullet$};
	\node at  (0,-.3) {$\scriptstyle a$};
	\node at  (.3,-.3) {$\scriptstyle b$};
	\node at  (1.2,-.3) {$\scriptstyle c$};
	\node at  (1.5,-.3) {$\scriptstyle d$};
	\end{tikzpicture} 
	=D 
	\\ \\ \\
	A =  
	\begin{tikzpicture}[scale=1,baseline=.0cm]
	\node at (-.3,.3) {$\scriptstyle I$}; \draw [thick,{Bar[width=3pt]}-{Bar[width=3pt]}] (0,.3)--(1.2,.3);   \node at (.3,.3) {$\bullet$};
	\node at (-.3,0) {$\scriptstyle J$};  \draw [thick,{Bar[width=3pt]}-{Bar[width=3pt]}] (.3,0)--(1.5,0);   \node at (.3,0) {$\bullet$};
	\node at  (0,-.3) {$\scriptstyle a$};
	\node at  (.3,-.3) {$\scriptstyle b$};
	\node at  (1.2,-.3) {$\scriptstyle c$};
	\node at  (1.5,-.3) {$\scriptstyle d$};
	\end{tikzpicture} 
	&\qquad\quad&
	\begin{tikzpicture}[scale=1,baseline=.0cm]
	\node at (2,.3) {$\scriptstyle I$}; \draw [thick,{Bar[width=3pt]}-{Bar[width=3pt]}] (0,.3)--(1.2,.3);   \node at (0,.3) {$\bullet$};
	\node at (2,0) {$\scriptstyle J$};  \draw [thick,{Bar[width=3pt]}-{Bar[width=3pt]}] (.3,0)--(1.5,0);   \node at (1.2,0) {$\bullet$};
	\node at  (0,-.3) {$\scriptstyle a$};
	\node at  (.3,-.3) {$\scriptstyle b$};
	\node at  (1.2,-.3) {$\scriptstyle c$};
	\node at  (1.5,-.3) {$\scriptstyle d$};
	\end{tikzpicture} 
	=B
\end{array}
$$
We have that $\pi_A<\pi_C$, $\pi_A<\pi_D$ and $\pi_B<\pi_D$ in the weak order. Proposition~\ref{prop:WeakToP} implies that $A<C$, $A<D$ and $B<D$ in the $P_\II$ order.
We claim that $B<C$  but this does not follow directly from the weak order since $\pi_B\not<\pi_C$. To show our claim, consider $\pi_E=adcbX$ and $\pi_F=adbcX$.
For any $K\in \II$ such that $a,d\not\in K$, we cannot have both $c,d\in K$, since this would imply that $K=[c,d]=I\cap J\not\in \II$, a contradiction.
This shows that $E=\Or_{\pi_E}=\Or_{\pi_F}=F$. Now we have $\pi_B<\pi_E$ and $\pi_F<\pi_C$ which gives $B<E=F<C$.
 
If the poset $P_\II$ is a lattice, then we can find a unique join $M= A \join B$. By definition $A\le M$ and $B\le M$. By property of join we also have $M\le C$ and $M\le D$ since both $C$ and $D$ are  greater than $A$ and $B$.
Let $\pi_M$ be any permutation such that $M=\Or_{\pi_M}$ and let  
   $$m=\pi\big(\min\{i:\pi_M(i)\in I\cup J\}\big).$$
If $m<b$, then $(m,I)\in M$ and $(b,I)\in A$ and Proposition~\ref{prop:sourceorder} implies that $A\not\le M$.
By a similar argument: $\big(b\le m<c \implies B\not\le M\big)$,  $\big(b< m\le c \implies M\not\le C\big)$, and $\big(c<m \implies M\not\le D\big)$. This is a contradiction to the existence of $m$.
\end{proof}


%%%%%%%%%%%%%%%%%%%%%%%%%%%%%%%%%%%%%%
%%%%%%%%%%%%%%%%%%%%%%%%%%%%%%%%%%%%%%
%%%%%%%%%%%%%%%%%%%%%%%%%%%%%%%%%%%%%%

\section*{Acknowledgments}

Thanks!
%%%%%%%%%%%%%%%%%%%%%%%%%%%%%%%%%%%%%%
%%%%%%%%%%%%%%%%%%%%%%%%%%%%%%%%%%%%%%
%%%%%%%%%%%%%%%%%%%%%%%%%%%%%%%%%%%%%%

\bibliographystyle{alpha}
\bibliography{IH_lattices}
\label{sec:biblio}

%%%%%%%%%%%%%%%%%%%%%%%%%%%%%%%%%%%%%%

\end{document}
